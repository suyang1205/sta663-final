
% Default to the notebook output style

    


% Inherit from the specified cell style.




    
\documentclass{article}

    
    
    \usepackage{graphicx} % Used to insert images
    \usepackage{adjustbox} % Used to constrain images to a maximum size 
    \usepackage{color} % Allow colors to be defined
    \usepackage{enumerate} % Needed for markdown enumerations to work
    \usepackage{geometry} % Used to adjust the document margins
    \usepackage{amsmath} % Equations
    \usepackage{amssymb} % Equations
    \usepackage[mathletters]{ucs} % Extended unicode (utf-8) support
    \usepackage[utf8x]{inputenc} % Allow utf-8 characters in the tex document
    \usepackage{fancyvrb} % verbatim replacement that allows latex
    \usepackage{grffile} % extends the file name processing of package graphics 
                         % to support a larger range 
    % The hyperref package gives us a pdf with properly built
    % internal navigation ('pdf bookmarks' for the table of contents,
    % internal cross-reference links, web links for URLs, etc.)
    \usepackage{hyperref}
    \usepackage{longtable} % longtable support required by pandoc >1.10
    \usepackage{booktabs}  % table support for pandoc > 1.12.2
    

    
    
    \definecolor{orange}{cmyk}{0,0.4,0.8,0.2}
    \definecolor{darkorange}{rgb}{.71,0.21,0.01}
    \definecolor{darkgreen}{rgb}{.12,.54,.11}
    \definecolor{myteal}{rgb}{.26, .44, .56}
    \definecolor{gray}{gray}{0.45}
    \definecolor{lightgray}{gray}{.95}
    \definecolor{mediumgray}{gray}{.8}
    \definecolor{inputbackground}{rgb}{.95, .95, .85}
    \definecolor{outputbackground}{rgb}{.95, .95, .95}
    \definecolor{traceback}{rgb}{1, .95, .95}
    % ansi colors
    \definecolor{red}{rgb}{.6,0,0}
    \definecolor{green}{rgb}{0,.65,0}
    \definecolor{brown}{rgb}{0.6,0.6,0}
    \definecolor{blue}{rgb}{0,.145,.698}
    \definecolor{purple}{rgb}{.698,.145,.698}
    \definecolor{cyan}{rgb}{0,.698,.698}
    \definecolor{lightgray}{gray}{0.5}
    
    % bright ansi colors
    \definecolor{darkgray}{gray}{0.25}
    \definecolor{lightred}{rgb}{1.0,0.39,0.28}
    \definecolor{lightgreen}{rgb}{0.48,0.99,0.0}
    \definecolor{lightblue}{rgb}{0.53,0.81,0.92}
    \definecolor{lightpurple}{rgb}{0.87,0.63,0.87}
    \definecolor{lightcyan}{rgb}{0.5,1.0,0.83}
    
    % commands and environments needed by pandoc snippets
    % extracted from the output of `pandoc -s`
    \DefineVerbatimEnvironment{Highlighting}{Verbatim}{commandchars=\\\{\}}
    % Add ',fontsize=\small' for more characters per line
    \newenvironment{Shaded}{}{}
    \newcommand{\KeywordTok}[1]{\textcolor[rgb]{0.00,0.44,0.13}{\textbf{{#1}}}}
    \newcommand{\DataTypeTok}[1]{\textcolor[rgb]{0.56,0.13,0.00}{{#1}}}
    \newcommand{\DecValTok}[1]{\textcolor[rgb]{0.25,0.63,0.44}{{#1}}}
    \newcommand{\BaseNTok}[1]{\textcolor[rgb]{0.25,0.63,0.44}{{#1}}}
    \newcommand{\FloatTok}[1]{\textcolor[rgb]{0.25,0.63,0.44}{{#1}}}
    \newcommand{\CharTok}[1]{\textcolor[rgb]{0.25,0.44,0.63}{{#1}}}
    \newcommand{\StringTok}[1]{\textcolor[rgb]{0.25,0.44,0.63}{{#1}}}
    \newcommand{\CommentTok}[1]{\textcolor[rgb]{0.38,0.63,0.69}{\textit{{#1}}}}
    \newcommand{\OtherTok}[1]{\textcolor[rgb]{0.00,0.44,0.13}{{#1}}}
    \newcommand{\AlertTok}[1]{\textcolor[rgb]{1.00,0.00,0.00}{\textbf{{#1}}}}
    \newcommand{\FunctionTok}[1]{\textcolor[rgb]{0.02,0.16,0.49}{{#1}}}
    \newcommand{\RegionMarkerTok}[1]{{#1}}
    \newcommand{\ErrorTok}[1]{\textcolor[rgb]{1.00,0.00,0.00}{\textbf{{#1}}}}
    \newcommand{\NormalTok}[1]{{#1}}
    
    % Define a nice break command that doesn't care if a line doesn't already
    % exist.
    \def\br{\hspace*{\fill} \\* }
    % Math Jax compatability definitions
    \def\gt{>}
    \def\lt{<}
    % Document parameters
    \title{final\_version}
    
    
    

    % Pygments definitions
    
\makeatletter
\def\PY@reset{\let\PY@it=\relax \let\PY@bf=\relax%
    \let\PY@ul=\relax \let\PY@tc=\relax%
    \let\PY@bc=\relax \let\PY@ff=\relax}
\def\PY@tok#1{\csname PY@tok@#1\endcsname}
\def\PY@toks#1+{\ifx\relax#1\empty\else%
    \PY@tok{#1}\expandafter\PY@toks\fi}
\def\PY@do#1{\PY@bc{\PY@tc{\PY@ul{%
    \PY@it{\PY@bf{\PY@ff{#1}}}}}}}
\def\PY#1#2{\PY@reset\PY@toks#1+\relax+\PY@do{#2}}

\expandafter\def\csname PY@tok@gd\endcsname{\def\PY@tc##1{\textcolor[rgb]{0.63,0.00,0.00}{##1}}}
\expandafter\def\csname PY@tok@gu\endcsname{\let\PY@bf=\textbf\def\PY@tc##1{\textcolor[rgb]{0.50,0.00,0.50}{##1}}}
\expandafter\def\csname PY@tok@gt\endcsname{\def\PY@tc##1{\textcolor[rgb]{0.00,0.27,0.87}{##1}}}
\expandafter\def\csname PY@tok@gs\endcsname{\let\PY@bf=\textbf}
\expandafter\def\csname PY@tok@gr\endcsname{\def\PY@tc##1{\textcolor[rgb]{1.00,0.00,0.00}{##1}}}
\expandafter\def\csname PY@tok@cm\endcsname{\let\PY@it=\textit\def\PY@tc##1{\textcolor[rgb]{0.25,0.50,0.50}{##1}}}
\expandafter\def\csname PY@tok@vg\endcsname{\def\PY@tc##1{\textcolor[rgb]{0.10,0.09,0.49}{##1}}}
\expandafter\def\csname PY@tok@m\endcsname{\def\PY@tc##1{\textcolor[rgb]{0.40,0.40,0.40}{##1}}}
\expandafter\def\csname PY@tok@mh\endcsname{\def\PY@tc##1{\textcolor[rgb]{0.40,0.40,0.40}{##1}}}
\expandafter\def\csname PY@tok@go\endcsname{\def\PY@tc##1{\textcolor[rgb]{0.53,0.53,0.53}{##1}}}
\expandafter\def\csname PY@tok@ge\endcsname{\let\PY@it=\textit}
\expandafter\def\csname PY@tok@vc\endcsname{\def\PY@tc##1{\textcolor[rgb]{0.10,0.09,0.49}{##1}}}
\expandafter\def\csname PY@tok@il\endcsname{\def\PY@tc##1{\textcolor[rgb]{0.40,0.40,0.40}{##1}}}
\expandafter\def\csname PY@tok@cs\endcsname{\let\PY@it=\textit\def\PY@tc##1{\textcolor[rgb]{0.25,0.50,0.50}{##1}}}
\expandafter\def\csname PY@tok@cp\endcsname{\def\PY@tc##1{\textcolor[rgb]{0.74,0.48,0.00}{##1}}}
\expandafter\def\csname PY@tok@gi\endcsname{\def\PY@tc##1{\textcolor[rgb]{0.00,0.63,0.00}{##1}}}
\expandafter\def\csname PY@tok@gh\endcsname{\let\PY@bf=\textbf\def\PY@tc##1{\textcolor[rgb]{0.00,0.00,0.50}{##1}}}
\expandafter\def\csname PY@tok@ni\endcsname{\let\PY@bf=\textbf\def\PY@tc##1{\textcolor[rgb]{0.60,0.60,0.60}{##1}}}
\expandafter\def\csname PY@tok@nl\endcsname{\def\PY@tc##1{\textcolor[rgb]{0.63,0.63,0.00}{##1}}}
\expandafter\def\csname PY@tok@nn\endcsname{\let\PY@bf=\textbf\def\PY@tc##1{\textcolor[rgb]{0.00,0.00,1.00}{##1}}}
\expandafter\def\csname PY@tok@no\endcsname{\def\PY@tc##1{\textcolor[rgb]{0.53,0.00,0.00}{##1}}}
\expandafter\def\csname PY@tok@na\endcsname{\def\PY@tc##1{\textcolor[rgb]{0.49,0.56,0.16}{##1}}}
\expandafter\def\csname PY@tok@nb\endcsname{\def\PY@tc##1{\textcolor[rgb]{0.00,0.50,0.00}{##1}}}
\expandafter\def\csname PY@tok@nc\endcsname{\let\PY@bf=\textbf\def\PY@tc##1{\textcolor[rgb]{0.00,0.00,1.00}{##1}}}
\expandafter\def\csname PY@tok@nd\endcsname{\def\PY@tc##1{\textcolor[rgb]{0.67,0.13,1.00}{##1}}}
\expandafter\def\csname PY@tok@ne\endcsname{\let\PY@bf=\textbf\def\PY@tc##1{\textcolor[rgb]{0.82,0.25,0.23}{##1}}}
\expandafter\def\csname PY@tok@nf\endcsname{\def\PY@tc##1{\textcolor[rgb]{0.00,0.00,1.00}{##1}}}
\expandafter\def\csname PY@tok@si\endcsname{\let\PY@bf=\textbf\def\PY@tc##1{\textcolor[rgb]{0.73,0.40,0.53}{##1}}}
\expandafter\def\csname PY@tok@s2\endcsname{\def\PY@tc##1{\textcolor[rgb]{0.73,0.13,0.13}{##1}}}
\expandafter\def\csname PY@tok@vi\endcsname{\def\PY@tc##1{\textcolor[rgb]{0.10,0.09,0.49}{##1}}}
\expandafter\def\csname PY@tok@nt\endcsname{\let\PY@bf=\textbf\def\PY@tc##1{\textcolor[rgb]{0.00,0.50,0.00}{##1}}}
\expandafter\def\csname PY@tok@nv\endcsname{\def\PY@tc##1{\textcolor[rgb]{0.10,0.09,0.49}{##1}}}
\expandafter\def\csname PY@tok@s1\endcsname{\def\PY@tc##1{\textcolor[rgb]{0.73,0.13,0.13}{##1}}}
\expandafter\def\csname PY@tok@sh\endcsname{\def\PY@tc##1{\textcolor[rgb]{0.73,0.13,0.13}{##1}}}
\expandafter\def\csname PY@tok@sc\endcsname{\def\PY@tc##1{\textcolor[rgb]{0.73,0.13,0.13}{##1}}}
\expandafter\def\csname PY@tok@sx\endcsname{\def\PY@tc##1{\textcolor[rgb]{0.00,0.50,0.00}{##1}}}
\expandafter\def\csname PY@tok@bp\endcsname{\def\PY@tc##1{\textcolor[rgb]{0.00,0.50,0.00}{##1}}}
\expandafter\def\csname PY@tok@c1\endcsname{\let\PY@it=\textit\def\PY@tc##1{\textcolor[rgb]{0.25,0.50,0.50}{##1}}}
\expandafter\def\csname PY@tok@kc\endcsname{\let\PY@bf=\textbf\def\PY@tc##1{\textcolor[rgb]{0.00,0.50,0.00}{##1}}}
\expandafter\def\csname PY@tok@c\endcsname{\let\PY@it=\textit\def\PY@tc##1{\textcolor[rgb]{0.25,0.50,0.50}{##1}}}
\expandafter\def\csname PY@tok@mf\endcsname{\def\PY@tc##1{\textcolor[rgb]{0.40,0.40,0.40}{##1}}}
\expandafter\def\csname PY@tok@err\endcsname{\def\PY@bc##1{\setlength{\fboxsep}{0pt}\fcolorbox[rgb]{1.00,0.00,0.00}{1,1,1}{\strut ##1}}}
\expandafter\def\csname PY@tok@kd\endcsname{\let\PY@bf=\textbf\def\PY@tc##1{\textcolor[rgb]{0.00,0.50,0.00}{##1}}}
\expandafter\def\csname PY@tok@ss\endcsname{\def\PY@tc##1{\textcolor[rgb]{0.10,0.09,0.49}{##1}}}
\expandafter\def\csname PY@tok@sr\endcsname{\def\PY@tc##1{\textcolor[rgb]{0.73,0.40,0.53}{##1}}}
\expandafter\def\csname PY@tok@mo\endcsname{\def\PY@tc##1{\textcolor[rgb]{0.40,0.40,0.40}{##1}}}
\expandafter\def\csname PY@tok@kn\endcsname{\let\PY@bf=\textbf\def\PY@tc##1{\textcolor[rgb]{0.00,0.50,0.00}{##1}}}
\expandafter\def\csname PY@tok@mi\endcsname{\def\PY@tc##1{\textcolor[rgb]{0.40,0.40,0.40}{##1}}}
\expandafter\def\csname PY@tok@gp\endcsname{\let\PY@bf=\textbf\def\PY@tc##1{\textcolor[rgb]{0.00,0.00,0.50}{##1}}}
\expandafter\def\csname PY@tok@o\endcsname{\def\PY@tc##1{\textcolor[rgb]{0.40,0.40,0.40}{##1}}}
\expandafter\def\csname PY@tok@kr\endcsname{\let\PY@bf=\textbf\def\PY@tc##1{\textcolor[rgb]{0.00,0.50,0.00}{##1}}}
\expandafter\def\csname PY@tok@s\endcsname{\def\PY@tc##1{\textcolor[rgb]{0.73,0.13,0.13}{##1}}}
\expandafter\def\csname PY@tok@kp\endcsname{\def\PY@tc##1{\textcolor[rgb]{0.00,0.50,0.00}{##1}}}
\expandafter\def\csname PY@tok@w\endcsname{\def\PY@tc##1{\textcolor[rgb]{0.73,0.73,0.73}{##1}}}
\expandafter\def\csname PY@tok@kt\endcsname{\def\PY@tc##1{\textcolor[rgb]{0.69,0.00,0.25}{##1}}}
\expandafter\def\csname PY@tok@ow\endcsname{\let\PY@bf=\textbf\def\PY@tc##1{\textcolor[rgb]{0.67,0.13,1.00}{##1}}}
\expandafter\def\csname PY@tok@sb\endcsname{\def\PY@tc##1{\textcolor[rgb]{0.73,0.13,0.13}{##1}}}
\expandafter\def\csname PY@tok@k\endcsname{\let\PY@bf=\textbf\def\PY@tc##1{\textcolor[rgb]{0.00,0.50,0.00}{##1}}}
\expandafter\def\csname PY@tok@se\endcsname{\let\PY@bf=\textbf\def\PY@tc##1{\textcolor[rgb]{0.73,0.40,0.13}{##1}}}
\expandafter\def\csname PY@tok@sd\endcsname{\let\PY@it=\textit\def\PY@tc##1{\textcolor[rgb]{0.73,0.13,0.13}{##1}}}

\def\PYZbs{\char`\\}
\def\PYZus{\char`\_}
\def\PYZob{\char`\{}
\def\PYZcb{\char`\}}
\def\PYZca{\char`\^}
\def\PYZam{\char`\&}
\def\PYZlt{\char`\<}
\def\PYZgt{\char`\>}
\def\PYZsh{\char`\#}
\def\PYZpc{\char`\%}
\def\PYZdl{\char`\$}
\def\PYZhy{\char`\-}
\def\PYZsq{\char`\'}
\def\PYZdq{\char`\"}
\def\PYZti{\char`\~}
% for compatibility with earlier versions
\def\PYZat{@}
\def\PYZlb{[}
\def\PYZrb{]}
\makeatother


    % Exact colors from NB
    \definecolor{incolor}{rgb}{0.0, 0.0, 0.5}
    \definecolor{outcolor}{rgb}{0.545, 0.0, 0.0}



    
    % Prevent overflowing lines due to hard-to-break entities
    \sloppy 
    % Setup hyperref package
    \hypersetup{
      breaklinks=true,  % so long urls are correctly broken across lines
      colorlinks=true,
      urlcolor=blue,
      linkcolor=darkorange,
      citecolor=darkgreen,
      }
    % Slightly bigger margins than the latex defaults
    
    \geometry{verbose,tmargin=1in,bmargin=1in,lmargin=1in,rmargin=1in}
    
    

    \begin{document}
    
    
    \maketitle
    
    

    
    \subsection{Final Project: Bayesian Hierarchical
Clustering}\label{final-project-bayesian-hierarchical-clustering}

\subsection{1. Background}\label{background}

Clustering is the task of grouping a set of objects in such a way that
objects in the same group (called a cluster) are more similar (in some
sense or another) to each other than to those in other groups
(clusters). Algorithms of clustering includes connectivity based
clustering (hierarchical clustering),centroid-based clustering(k-means
clustering), distribution-based clustering, and density-based
clustering. We mainly focus on hierarchical clustering.

\subsubsection{1.1 Traditional Hierarchical
Clustering}\label{traditional-hierarchical-clustering}

The traditional method for hierarchically clustering is a bottomup
agglomerative algorithm. It starts with each data point assigned to its
own cluster and iteratively merges the two closest clusters together
until all the data belongs to a single cluster. The nearest pair of
clusters is chosen based on a given distance measure (e.g.~Euclidean
distance between cluster means, or distance between nearest points).

\subsubsection{1.2 Limitations of Traditional Hierarchical
Clustering}\label{limitations-of-traditional-hierarchical-clustering}

The tradtional hierarchical clustering method has many limitations: -
The algorithm provides no guide to choosing the ``correct'' number of
clusters or the level at which to prune the tree. - It is often difficult
to know which distance metric to choose, especially for structured data
such as images or sequences. - The traditional algorithm does not define
a probabilistic model of the data, so it is hard to ask how ``good''a
clustering is, to compare to other models, to make predictions and
cluster new data into an existing hierarchy

\subsubsection{1.3 Bayesian Hierarchical
Clustering}\label{bayesian-hierarchical-clustering}

To deal with those problems, we come up with a novel algorithm for
agglomerative hierarchical clustering using bayesian probabilistic
model. Bayesian hierarchical clustering algorithm uses marginal
likelihoods to decide which clusters to merge and to avoid overfitting.
Basically it asks what the probability is that all the data in a
potential merge were generated from the same mixture component, and
compares this to exponentially many hypotheses at lower levels of the
tree.

    \begin{longtable}[c]{@{}l@{}}
\toprule\addlinespace
2. Algorithm and Implementation
\\\addlinespace
\bottomrule
\end{longtable}

\subsubsection{2.1 Theories Behind the
Algorithm}\label{theories-behind-the-algorithm}

The main difference between traditional hierarchical clustering method
and bayesian hierarchical clustering is taht bayesian hierarchical
clustering uses a statistical hypothesis test to choose which clusters
to merge.

In considering each merge, two hypotheses are compared. The first
hypothesis, which we will denote $H_1^k$ is that all the data in cluster
$D_k$ were in fact generated independently and identically from the same
probabilistic model, $p(x|θ)$ with unknown parameters $\theta$. To
evaluate the probability of the data under this hypothesis we need to
specify some prior over the parameters of the model, $p(\theta|\beta)$
with hyperparameters $\beta$. We now have the ingredients to compute the
probability of the data $D_k$ under $H_1^k$:

\[p(D_k|H_1^k)= \int{p(D_k|\theta)p(\theta|\beta) d\theta}=\int{[\prod_{x^{(i)}\in D_k}p(x^{(i)}|\theta)]p(\theta|\beta) d\theta} \]

This calculates the probability that all the data in $D_k$ were
generated from the same parameter values assuming a model of the form
$p(x|\theta)$. This is a natural model-based criterion for measuring how
well the data fit into one cluster.

The alternative hypothesis to $H_1^k$ would be that the data in $D_k$
has two or more clusters in it(the probability of a data set under a
tree (e.g. $p(D_i|T_i)$) is defined below.):
\[p(D_k|H_2^k)=p(D_i|T_i)p(D_j|T_j)\]

Combining the probability of the data under hypotheses $H_1^k$ and
$H_2^k$, weighted by the prior that all points in $D_k$ belong to one
cluster, $\pi_k= p(H_1^k)$, we obtain the marginal probability of the
data in tree $T_k$:
\[p(D_k|T_k)=\pi_k p(D_k|H_1^k)+(1-\pi_k)p(D_i|T_i)p(D_j|T_j)\]

Therefore, we can derive the posterior probability of the merged
hypothesis $r_k$ using Bayes rule:
\[r_k=\frac{\pi_k p(D_k|H_1^k)}{\pi_k p(D_k|H_1^k)+(1-\pi_k)p(D_i|T_i)p(D_j|T_j)}\]

    \subsubsection{2.2 Bayesian Hierarchical Clustering
Algorithm}\label{bayesian-hierarchical-clustering-algorithm}

The general algorithm is shown below:

\textbf{input:} data $D={x^{(1)},...x^{(n)}}$, model $p(x|\theta)$,
prior $p(\theta|\beta)$

\textbf{inialize:} number of clusters $c=n$, and $D_i={x^{(i)}}$ for
$i=1,2,...n$

\textbf{while} $c>1$ \textbf{do:}

 Find the pair $D_i$ and $D_j$ with the highest probability of the
merged hypothesis $r_k$

 Merge $D_k\leftarrow D_i \bigcup D_j$, $T_k\leftarrow (T_i, T_j)$

 Delete $D_i$ and $D_j$, $c\leftarrow c-1$

\textbf{end while}

\textbf{output:} Bayesian mixture model where each tree node is a
mixture component

The tree can be cut at points where $r_k < 0.5$

    \subsubsection{2.3 Implementation in
Python}\label{implementation-in-python}

\paragraph{2.3.1 Data Structure}\label{data-structure}

In the clustering process, in order to store the information for each
cluster, I define a \textbf{class} called \textbf{``bicluster''} to
record the properties of each cluster(i.e.~the data structure for
cluster is \textbf{class}). The properties for the class
\textbf{bicluster} includes:

\begin{itemize}
\item
  left: one of the two clusters in the new cluster after merging
\item
  right: the other one of the two clusters in the new cluster after
  merging
\item
  point: all data points in the cluster
\item
  id: id for each cluster
\item
  probability: the marginal probability of the data in the tree
\item
  d: a parameter used in calculating the prior that all data points in
  $D_k$ belongs to one cluster
\end{itemize}

    \begin{Verbatim}[commandchars=\\\{\}]
{\color{incolor}In [{\color{incolor}2}]:} \PY{c}{\PYZsh{}define a class called \PYZdq{}bicluster\PYZdq{} to represent cluster and its properties}
        \PY{k}{class} \PY{n+nc}{bicluster}\PY{p}{:}
            \PY{k}{def} \PY{n+nf}{\PYZus{}\PYZus{}init\PYZus{}\PYZus{}}\PY{p}{(}\PY{n+nb+bp}{self}\PY{p}{,} \PY{n}{point}\PY{p}{,} \PY{n}{left}\PY{o}{=}\PY{n+nb+bp}{None}\PY{p}{,}\PY{n}{right}\PY{o}{=}\PY{n+nb+bp}{None}\PY{p}{,}\PY{n}{probability}\PY{o}{=}\PY{n+nb+bp}{None}\PY{p}{,}\PY{n}{d}\PY{o}{=}\PY{n+nb+bp}{None}\PY{p}{,}\PY{n+nb}{id}\PY{o}{=}\PY{n+nb+bp}{None}\PY{p}{)}\PY{p}{:}
                \PY{n+nb+bp}{self}\PY{o}{.}\PY{n}{left} \PY{o}{=} \PY{n}{left}
                \PY{n+nb+bp}{self}\PY{o}{.}\PY{n}{right} \PY{o}{=} \PY{n}{right}  
                \PY{n+nb+bp}{self}\PY{o}{.}\PY{n}{point} \PY{o}{=} \PY{n}{point}   
                \PY{n+nb+bp}{self}\PY{o}{.}\PY{n}{id} \PY{o}{=} \PY{n+nb}{id}     
                \PY{n+nb+bp}{self}\PY{o}{.}\PY{n}{probability} \PY{o}{=} \PY{n}{probability}
                \PY{n+nb+bp}{self}\PY{o}{.}\PY{n}{d}\PY{o}{=}\PY{n}{d}
\end{Verbatim}

    \paragraph{2.3.2 Input}\label{input}

The input for the clustering function contains two part: a \textbf{data
set} used for clustering and a \textbf{function} which calculates the
probability that all data in $D_k$ are generated independently and
identically from a same distribution(i.e. $P(D_K|H_1^k)$). The choice
for the distribution and prior are in the following part.

    \paragraph{2.3.3 Choice fot the probablistic model and corresponding
prior}\label{choice-fot-the-probablistic-model-and-corresponding-prior}

For the probablistic model and prior, I choose
\textbf{Normal-Inverse-Wishart} and \textbf{Dirichlet-Multinomial} in
the implememtation. Since they are both conjugate priors, I can
calculate the integral:

\begin{itemize}
\itemsep1pt\parskip0pt\parsep0pt
\item
  Normal-Inverse-Wishart(multinormal data):
\end{itemize}

\[\Sigma \sim IW(\Sigma_0^{-1}, \nu_0)\]
\[\mu \sim N(\mu_0, \Sigma/\kappa_0)\]

The integral $p(D_k|H_1^k)= \int{p(D_k|\theta)p(\theta|\beta) d\theta}$
is:

\[p(D_k|H_1^k)= \frac{1}{\pi^{nd/2}}\frac{{|\Lambda_0|}^{\nu_0/2}}{{|\Lambda_n|}^{\nu_n/2}} (\kappa_0/\kappa_n)^{d/2} \frac{\Gamma_d(\nu_n/2)}{\Gamma_d(\nu_0/2)}\]

where
$\mu_n=\frac{\kappa_0}{\kappa_0+n}\mu_0+\frac{n}{\kappa_0+n}\bar{x}$,
$\kappa_n=\kappa_0+n$, $\nu_n=\nu_0+n$,
$\Lambda_n=\Lambda_0+\sum_i ^{N} (x_i-\bar{x})(x_i-\bar{x})'+\frac{\kappa_0}{\kappa_0+n}(\bar{x}-\mu_0)(\bar{x}-\mu_0)'$

    \begin{Verbatim}[commandchars=\\\{\}]
{\color{incolor}In [{\color{incolor}3}]:} \PY{c}{\PYZsh{}function to calculate the marginal probability under the H1 hypothesis using Normal\PYZhy{}Inverse\PYZhy{}Wishart distribution}
        \PY{k+kn}{import} \PY{n+nn}{numpy} \PY{k+kn}{as} \PY{n+nn}{np}
        \PY{k+kn}{from} \PY{n+nn}{scipy.stats} \PY{k+kn}{import} \PY{n}{multivariate\PYZus{}normal}
        \PY{k+kn}{import} \PY{n+nn}{scipy}
        \PY{k+kn}{from} \PY{n+nn}{scipy.special} \PY{k+kn}{import} \PY{n}{gamma} 
        \PY{k+kn}{import} \PY{n+nn}{random}
        \PY{k+kn}{from} \PY{n+nn}{math} \PY{k+kn}{import} \PY{n}{sqrt}  
        
        \PY{k}{def} \PY{n+nf}{marginal\PYZus{}likelihood\PYZus{}NIW}\PY{p}{(}\PY{n}{points}\PY{p}{)}\PY{p}{:}
            \PY{n}{n}\PY{o}{=}\PY{n+nb}{len}\PY{p}{(}\PY{n}{points}\PY{p}{)}
            \PY{n}{p}\PY{o}{=}\PY{n+nb}{len}\PY{p}{(}\PY{n}{points}\PY{p}{[}\PY{l+m+mi}{0}\PY{p}{]}\PY{p}{)}
            \PY{n}{mean\PYZus{}data} \PY{o}{=} \PY{n}{np}\PY{o}{.}\PY{n}{mean}\PY{p}{(}\PY{n}{points}\PY{p}{,} \PY{n}{axis}\PY{o}{=}\PY{l+m+mi}{0}\PY{p}{)}
            \PY{n}{sum\PYZus{}squares} \PY{o}{=} \PY{n}{np}\PY{o}{.}\PY{n}{sum}\PY{p}{(}\PY{p}{[}\PY{n}{np}\PY{o}{.}\PY{n}{array}\PY{p}{(}\PY{n}{np}\PY{o}{.}\PY{n}{matrix}\PY{p}{(}\PY{n}{x} \PY{o}{\PYZhy{}} \PY{n}{mean\PYZus{}data}\PY{p}{)}\PY{o}{.}\PY{n}{T} \PY{o}{*} \PY{n}{np}\PY{o}{.}\PY{n}{matrix}\PY{p}{(}\PY{n}{x} \PY{o}{\PYZhy{}} \PY{n}{mean\PYZus{}data}\PY{p}{)}\PY{p}{)} \PY{k}{for} \PY{n}{x} \PY{o+ow}{in} \PY{n}{points}\PY{p}{]}\PY{p}{,} \PY{n}{axis}\PY{o}{=}\PY{l+m+mi}{0}\PY{p}{)}
            \PY{c}{\PYZsh{}parameters for the prior distribution}
            \PY{n}{k0}\PY{o}{=}\PY{l+m+mf}{3.0}
            \PY{n}{v0}\PY{o}{=}\PY{l+m+mi}{10}
            \PY{n}{sigma0}\PY{o}{=}\PY{n}{np}\PY{o}{.}\PY{n}{eye}\PY{p}{(}\PY{n}{p}\PY{p}{)}
            \PY{n}{mu0}\PY{o}{=}\PY{p}{[}\PY{l+m+mi}{0}\PY{p}{]}\PY{o}{*}\PY{n}{p}
            \PY{c}{\PYZsh{}parameters for the posterior distribution}
            \PY{n}{kn}\PY{o}{=}\PY{n}{k0}\PY{o}{+}\PY{n}{n}
            \PY{n}{vn}\PY{o}{=}\PY{n}{v0}\PY{o}{+}\PY{n}{n}
            \PY{n}{sigman}\PY{o}{=}\PY{n}{sigma0}\PY{o}{+}\PY{n}{sum\PYZus{}squares}\PY{o}{+}\PY{p}{(}\PY{n+nb}{float}\PY{p}{(}\PY{n}{k0}\PY{o}{*}\PY{n}{n}\PY{p}{)}\PY{o}{/}\PY{p}{(}\PY{n}{k0}\PY{o}{+}\PY{n}{n}\PY{p}{)}\PY{p}{)}\PY{o}{*}\PY{n}{np}\PY{o}{.}\PY{n}{dot}\PY{p}{(}\PY{p}{(}\PY{n}{mean\PYZus{}data}\PY{o}{\PYZhy{}}\PY{n}{mu0}\PY{p}{)}\PY{o}{.}\PY{n}{reshape}\PY{p}{(}\PY{n}{p}\PY{p}{,}\PY{l+m+mi}{1}\PY{p}{)}\PY{p}{,}\PY{p}{(}\PY{n}{mean\PYZus{}data}\PY{o}{\PYZhy{}}\PY{n}{mu0}\PY{p}{)}\PY{o}{.}\PY{n}{reshape}\PY{p}{(}\PY{l+m+mi}{1}\PY{p}{,}\PY{n}{p}\PY{p}{)}\PY{p}{)}
            \PY{n}{gammadn}\PY{o}{=}\PY{n+nb}{reduce}\PY{p}{(}\PY{k}{lambda} \PY{n}{x}\PY{p}{,}\PY{n}{y}\PY{p}{:} \PY{n}{x}\PY{o}{*}\PY{n}{y}\PY{p}{,} \PY{n+nb}{map}\PY{p}{(}\PY{k}{lambda} \PY{n}{x}\PY{p}{:}\PY{n}{gamma}\PY{p}{(}\PY{p}{(}\PY{n}{vn}\PY{o}{+}\PY{l+m+mi}{1}\PY{o}{\PYZhy{}}\PY{n}{x}\PY{p}{)}\PY{o}{*}\PY{l+m+mf}{0.5}\PY{p}{)}\PY{p}{,}\PY{n+nb}{range}\PY{p}{(}\PY{l+m+mi}{1}\PY{p}{,}\PY{n}{p}\PY{o}{+}\PY{l+m+mi}{1}\PY{p}{)}\PY{p}{)}\PY{p}{)}
            \PY{n}{gammad0}\PY{o}{=}\PY{n+nb}{reduce}\PY{p}{(}\PY{k}{lambda} \PY{n}{x}\PY{p}{,}\PY{n}{y}\PY{p}{:} \PY{n}{x}\PY{o}{*}\PY{n}{y}\PY{p}{,} \PY{n+nb}{map}\PY{p}{(}\PY{k}{lambda} \PY{n}{x}\PY{p}{:}\PY{n}{gamma}\PY{p}{(}\PY{p}{(}\PY{n}{v0}\PY{o}{+}\PY{l+m+mi}{1}\PY{o}{\PYZhy{}}\PY{n}{x}\PY{p}{)}\PY{o}{*}\PY{l+m+mf}{0.5}\PY{p}{)}\PY{p}{,}\PY{n+nb}{range}\PY{p}{(}\PY{l+m+mi}{1}\PY{p}{,}\PY{n}{p}\PY{o}{+}\PY{l+m+mi}{1}\PY{p}{)}\PY{p}{)}\PY{p}{)}
            \PY{c}{\PYZsh{}calculate the marginal probability}
            \PY{n}{mar\PYZus{}like}\PY{o}{=}\PY{n}{np}\PY{o}{.}\PY{n}{pi}\PY{o}{*}\PY{o}{*}\PY{p}{(}\PY{o}{\PYZhy{}}\PY{n}{p}\PY{o}{*}\PY{n}{n}\PY{o}{*}\PY{l+m+mf}{0.5}\PY{p}{)}\PY{o}{*}\PY{p}{(}\PY{n}{np}\PY{o}{.}\PY{n}{linalg}\PY{o}{.}\PY{n}{det}\PY{p}{(}\PY{n}{sigma0}\PY{p}{)}\PY{p}{)}\PY{o}{*}\PY{o}{*}\PY{p}{(}\PY{n}{v0}\PY{o}{*}\PY{l+m+mf}{0.5}\PY{p}{)}\PY{o}{*}\PY{p}{(}\PY{n}{k0}\PY{o}{/}\PY{n}{kn}\PY{p}{)}\PY{o}{*}\PY{o}{*}\PY{p}{(}\PY{l+m+mf}{0.5}\PY{o}{*}\PY{n}{p}\PY{p}{)}\PY{o}{*}\PY{n}{gammadn}\PY{o}{/}\PY{p}{(}\PY{n+nb}{float}\PY{p}{(}\PY{n}{gammad0}\PY{p}{)}\PY{o}{*}\PY{p}{(}\PY{n}{np}\PY{o}{.}\PY{n}{linalg}\PY{o}{.}\PY{n}{det}\PY{p}{(}\PY{n}{sigman}\PY{p}{)}\PY{p}{)}\PY{o}{*}\PY{o}{*}\PY{p}{(}\PY{n}{vn}\PY{o}{*}\PY{l+m+mf}{0.5}\PY{p}{)}\PY{p}{)}
            
            \PY{k}{return} \PY{n}{mar\PYZus{}like}
\end{Verbatim}

    \begin{itemize}
\itemsep1pt\parskip0pt\parsep0pt
\item
  Dirichlet-Multinomial(multinomial data):
  \[p\sim \frac{\Gamma(\sum \alpha_i)}{\prod_i^{k} \Gamma(\alpha_i)} \prod_i^{k}p_i^{\alpha_i-1}\]
\end{itemize}

The integral $p(D_k|H_1^k)= \int{p(D_k|\theta)p(\theta|\beta) d\theta}$
is:

\[p(D_k|H_1^k)=\prod_i [\frac{k!}{\prod_c x_i,c!} \frac{\Gamma(\alpha_0)}{\prod_c \Gamma(\alpha_c)} \frac{\prod_c \Gamma(\alpha_c+x_i,c)}{\Gamma(\alpha_0+k)}]\]

where $\alpha_0=\sum \alpha_i$

    \begin{Verbatim}[commandchars=\\\{\}]
{\color{incolor}In [{\color{incolor}4}]:} \PY{c}{\PYZsh{}function to calculate the marginal probability under the H1 hypothesis using Dirichlet\PYZhy{}Multinomial distribution}
        \PY{k}{def} \PY{n+nf}{marginal\PYZus{}likelihood\PYZus{}DW}\PY{p}{(}\PY{n}{points}\PY{p}{)}\PY{p}{:}
            \PY{n}{p}\PY{o}{=}\PY{n+nb}{len}\PY{p}{(}\PY{n}{points}\PY{p}{[}\PY{l+m+mi}{0}\PY{p}{]}\PY{p}{)}
            \PY{n}{n}\PY{o}{=}\PY{n+nb}{len}\PY{p}{(}\PY{n}{points}\PY{p}{)}
            \PY{c}{\PYZsh{}parameters for the prior distribution}
            \PY{n}{alpha}\PY{o}{=}\PY{p}{[}\PY{l+m+mf}{1.0}\PY{o}{/}\PY{n}{p}\PY{p}{]}\PY{o}{*}\PY{n}{p}
            \PY{n}{alpha0}\PY{o}{=}\PY{n}{np}\PY{o}{.}\PY{n}{sum}\PY{p}{(}\PY{n}{alpha}\PY{p}{)}
            
            \PY{k}{def} \PY{n+nf}{temp}\PY{p}{(}\PY{n}{yi}\PY{p}{)}\PY{p}{:}
                \PY{n+nb}{sum}\PY{o}{=}\PY{l+m+mf}{1.0}
                \PY{k}{for} \PY{n}{i} \PY{o+ow}{in} \PY{n+nb}{range}\PY{p}{(}\PY{l+m+mi}{0}\PY{p}{,}\PY{n}{p}\PY{p}{)}\PY{p}{:}
                    \PY{n+nb}{sum}\PY{o}{=}\PY{n+nb}{sum}\PY{o}{*}\PY{n}{gamma}\PY{p}{(}\PY{n}{alpha}\PY{p}{[}\PY{n}{i}\PY{p}{]}\PY{o}{+}\PY{n}{yi}\PY{p}{[}\PY{n}{i}\PY{p}{]}\PY{p}{)}\PY{o}{/}\PY{p}{(}\PY{n}{gamma}\PY{p}{(}\PY{n}{yi}\PY{p}{[}\PY{n}{i}\PY{p}{]}\PY{o}{+}\PY{l+m+mi}{1}\PY{p}{)}\PY{o}{*}\PY{n}{gamma}\PY{p}{(}\PY{n}{alpha}\PY{p}{[}\PY{n}{i}\PY{p}{]}\PY{p}{)}\PY{p}{)}
                \PY{k}{return} \PY{n+nb}{sum}
            \PY{c}{\PYZsh{}calculate the marginal probability}
            \PY{n}{mar\PYZus{}like}\PY{o}{=}\PY{n+nb}{reduce}\PY{p}{(}\PY{k}{lambda} \PY{n}{x}\PY{p}{,}\PY{n}{y}\PY{p}{:} \PY{n}{x}\PY{o}{*}\PY{n}{y}\PY{p}{,} \PY{n+nb}{map}\PY{p}{(}\PY{n}{temp}\PY{p}{,}\PY{n}{points}\PY{p}{)}\PY{p}{)}\PY{o}{*}\PY{p}{(}\PY{n}{gamma}\PY{p}{(}\PY{n}{p}\PY{o}{+}\PY{l+m+mi}{1}\PY{p}{)}\PY{o}{*}\PY{n}{gamma}\PY{p}{(}\PY{n}{alpha0}\PY{p}{)}\PY{o}{/}\PY{n+nb}{float}\PY{p}{(}\PY{n}{gamma}\PY{p}{(}\PY{n}{alpha0}\PY{o}{+}\PY{n}{p}\PY{p}{)}\PY{p}{)}\PY{p}{)}\PY{o}{*}\PY{o}{*}\PY{n}{n}
            
            \PY{k}{return} \PY{n}{mar\PYZus{}like}
\end{Verbatim}

    \paragraph{2.3.4 Function for Bayesian Hierarchical
Clustering}\label{function-for-bayesian-hierarchical-clustering}

    \begin{Verbatim}[commandchars=\\\{\}]
{\color{incolor}In [{\color{incolor}5}]:} \PY{c}{\PYZsh{}function used to store the clustering result}
        \PY{k}{def} \PY{n+nf}{yezi}\PY{p}{(}\PY{n}{clust}\PY{p}{)}\PY{p}{:}
            \PY{k}{if} \PY{n}{clust}\PY{o}{.}\PY{n}{left} \PY{o}{==} \PY{n+nb+bp}{None} \PY{o+ow}{and} \PY{n}{clust}\PY{o}{.}\PY{n}{right} \PY{o}{==} \PY{n+nb+bp}{None} \PY{p}{:}
                \PY{k}{return} \PY{p}{[}\PY{n}{clust}\PY{o}{.}\PY{n}{id}\PY{p}{]}
            \PY{k}{return} \PY{n}{yezi}\PY{p}{(}\PY{n}{clust}\PY{o}{.}\PY{n}{left}\PY{p}{)} \PY{o}{+} \PY{n}{yezi}\PY{p}{(}\PY{n}{clust}\PY{o}{.}\PY{n}{right}\PY{p}{)}
         
            
        \PY{c}{\PYZsh{}cluster function}
        \PY{c}{\PYZsh{}parameter \PYZdq{}data\PYZdq{} is the data sset we want to cluster}
        \PY{c}{\PYZsh{}parameter \PYZdq{}function\PYZdq{} is the function to calculate the marginal probability, which involves the choice of probablistic model and prior }
        \PY{k}{def} \PY{n+nf}{bcluster}\PY{p}{(}\PY{n}{data}\PY{p}{,}\PY{n}{function}\PY{p}{)} \PY{p}{:}
            \PY{n}{dim}\PY{o}{=}\PY{n+nb}{len}\PY{p}{(}\PY{n}{data}\PY{p}{[}\PY{l+m+mi}{0}\PY{p}{]}\PY{p}{)}\PY{c}{\PYZsh{}get the dimension of the data points}
            \PY{n}{alpha}\PY{o}{=}\PY{l+m+mi}{3}
            \PY{n}{biclusters} \PY{o}{=} \PY{p}{[} \PY{n}{bicluster}\PY{p}{(}\PY{n}{point} \PY{o}{=} \PY{p}{[}\PY{n}{data}\PY{p}{[}\PY{n}{i}\PY{p}{]}\PY{p}{]}\PY{p}{,} \PY{n+nb}{id} \PY{o}{=} \PY{n}{i} \PY{p}{,}\PY{n}{probability}\PY{o}{=}\PY{l+m+mf}{0.0000001}\PY{p}{,}\PY{n}{d}\PY{o}{=}\PY{n}{alpha}\PY{p}{)} \PY{k}{for} \PY{n}{i} \PY{o+ow}{in} \PY{n+nb}{range}\PY{p}{(}\PY{n+nb}{len}\PY{p}{(}\PY{n}{data}\PY{p}{)}\PY{p}{)}\PY{p}{]} \PY{c}{\PYZsh{}initialize: each point is a cluster}
            \PY{n}{flag} \PY{o}{=} \PY{n+nb+bp}{None}\PY{p}{;}\PY{c}{\PYZsh{}record the id of the pair of clusters when merging}
            \PY{n}{currentclusted} \PY{o}{=} \PY{o}{\PYZhy{}}\PY{l+m+mi}{1} \PY{c}{\PYZsh{}id for the new cluster}
            
            \PY{c}{\PYZsh{}when there is only one data point in the data set}
            \PY{k}{if} \PY{n+nb}{len}\PY{p}{(}\PY{n}{biclusters}\PY{p}{)} \PY{o}{==} \PY{l+m+mi}{1}\PY{p}{:}
                \PY{n}{clusters} \PY{o}{=} \PY{p}{[}\PY{n}{yezi}\PY{p}{(}\PY{n}{biclusters}\PY{p}{[}\PY{n}{i}\PY{p}{]}\PY{p}{)} \PY{k}{for} \PY{n}{i} \PY{o+ow}{in} \PY{n+nb}{range}\PY{p}{(}\PY{n+nb}{len}\PY{p}{(}\PY{n}{biclusters}\PY{p}{)}\PY{p}{)}\PY{p}{]} 
                \PY{k}{return} \PY{n}{biclusters}\PY{p}{,}\PY{n}{clusters}
            
            \PY{c}{\PYZsh{}when there are more than one data points in the data set}
            \PY{k}{while}\PY{p}{(}\PY{n+nb}{len}\PY{p}{(}\PY{n}{biclusters}\PY{p}{)} \PY{o}{\PYZgt{}} \PY{l+m+mi}{1}\PY{p}{)} \PY{p}{:} 
                \PY{n}{max\PYZus{}prob} \PY{o}{=} \PY{l+m+mi}{0}\PY{p}{;} \PY{c}{\PYZsh{}initializa the maximum r\PYZus{}k}
                \PY{n}{biclusters\PYZus{}len} \PY{o}{=} \PY{n+nb}{len}\PY{p}{(}\PY{n}{biclusters}\PY{p}{)}\PY{c}{\PYZsh{}number of the current clusters}
                
                \PY{c}{\PYZsh{}find the pair of clusters that has the largest probability of merged hypothesis}
                \PY{k}{for} \PY{n}{i} \PY{o+ow}{in} \PY{n+nb}{range}\PY{p}{(}\PY{n}{biclusters\PYZus{}len}\PY{o}{\PYZhy{}}\PY{l+m+mi}{1}\PY{p}{)} \PY{p}{:}
                    \PY{k}{for} \PY{n}{j} \PY{o+ow}{in} \PY{n+nb}{range}\PY{p}{(}\PY{n}{i} \PY{o}{+} \PY{l+m+mi}{1}\PY{p}{,} \PY{n}{biclusters\PYZus{}len}\PY{p}{)} \PY{p}{:}
                        
                        \PY{n}{temp\PYZus{}cluster\PYZus{}points}\PY{o}{=} \PY{n}{biclusters}\PY{p}{[}\PY{n}{i}\PY{p}{]}\PY{o}{.}\PY{n}{point} \PY{o}{+} \PY{n}{biclusters}\PY{p}{[}\PY{n}{j}\PY{p}{]}\PY{o}{.}\PY{n}{point}\PY{c}{\PYZsh{}combine the two clusters into one cluster}
                        \PY{n}{P\PYZus{}H1}\PY{o}{=}\PY{n}{function}\PY{p}{(}\PY{n}{temp\PYZus{}cluster\PYZus{}points}\PY{p}{)}\PY{c}{\PYZsh{}calculate the  probability that the points of the two clusters under H\PYZus{}1 hypothesis }
                        \PY{c}{\PYZsh{}update the prior that all data points belong to one cluster}
                        \PY{n}{pi}\PY{o}{=}\PY{n+nb}{float}\PY{p}{(}\PY{n}{scipy}\PY{o}{.}\PY{n}{misc}\PY{o}{.}\PY{n}{factorial}\PY{p}{(}\PY{n+nb}{len}\PY{p}{(}\PY{n}{temp\PYZus{}cluster\PYZus{}points}\PY{p}{)}\PY{o}{\PYZhy{}}\PY{l+m+mi}{1}\PY{p}{)}\PY{p}{)}\PY{o}{*}\PY{n}{alpha}\PY{o}{/}\PY{p}{(}\PY{n+nb}{float}\PY{p}{(}\PY{n}{scipy}\PY{o}{.}\PY{n}{misc}\PY{o}{.}\PY{n}{factorial}\PY{p}{(}\PY{n+nb}{len}\PY{p}{(}\PY{n}{temp\PYZus{}cluster\PYZus{}points}\PY{p}{)}\PY{o}{\PYZhy{}}\PY{l+m+mi}{1}\PY{p}{)}\PY{p}{)}\PY{o}{*}\PY{n}{alpha}\PY{o}{+}\PY{n}{biclusters}\PY{p}{[}\PY{n}{i}\PY{p}{]}\PY{o}{.}\PY{n}{d}\PY{o}{*}\PY{n}{biclusters}\PY{p}{[}\PY{n}{j}\PY{p}{]}\PY{o}{.}\PY{n}{d}\PY{p}{)}
                        \PY{n}{marginal\PYZus{}prob}\PY{o}{=}\PY{n}{pi}\PY{o}{*}\PY{n}{P\PYZus{}H1}\PY{o}{+}\PY{p}{(}\PY{l+m+mi}{1}\PY{o}{\PYZhy{}}\PY{n}{pi}\PY{p}{)}\PY{o}{*}\PY{n}{biclusters}\PY{p}{[}\PY{n}{i}\PY{p}{]}\PY{o}{.}\PY{n}{probability}\PY{o}{*}\PY{n}{biclusters}\PY{p}{[}\PY{n}{j}\PY{p}{]}\PY{o}{.}\PY{n}{probability}\PY{c}{\PYZsh{}calculate the marginal probabilty of data in tree T\PYZus{}k}
                        \PY{n}{r} \PY{o}{=} \PY{n}{pi}\PY{o}{*}\PY{n}{P\PYZus{}H1}\PY{o}{/}\PY{n}{marginal\PYZus{}prob}\PY{c}{\PYZsh{}calculate the posterior probability of merged hypothesis using bayes rule}
                        \PY{k}{if} \PY{n}{r} \PY{o}{\PYZgt{}} \PY{n}{max\PYZus{}prob} \PY{p}{:}
                            \PY{n}{max\PYZus{}prob} \PY{o}{=} \PY{n}{r}
                            \PY{n}{flag} \PY{o}{=} \PY{p}{(}\PY{n}{i}\PY{p}{,}\PY{n}{j}\PY{p}{)}
                            
                \PY{c}{\PYZsh{}when to stop clustering     }
                \PY{k}{if} \PY{n}{max\PYZus{}prob}\PY{o}{\PYZlt{}}\PY{l+m+mf}{0.5}\PY{p}{:} 
                    \PY{k}{break}
                    
                \PY{n}{bic1}\PY{p}{,}\PY{n}{bic2} \PY{o}{=} \PY{n}{flag} \PY{c}{\PYZsh{}update the flag}
                
                \PY{n}{newpoint} \PY{o}{=} \PY{n}{biclusters}\PY{p}{[}\PY{n}{bic1}\PY{p}{]}\PY{o}{.}\PY{n}{point} \PY{o}{+} \PY{n}{biclusters}\PY{p}{[}\PY{n}{bic2}\PY{p}{]}\PY{o}{.}\PY{n}{point} \PY{c}{\PYZsh{}combine the points of two clusters into the new cluster}
                \PY{n}{P\PYZus{}H1}\PY{o}{=}\PY{n}{function}\PY{p}{(}\PY{n}{newpoint}\PY{p}{)}
                \PY{c}{\PYZsh{}calculate the corresponding marginal probability for the new cluster}
                \PY{n}{newprob}\PY{o}{=}\PY{n}{pi}\PY{o}{*}\PY{n}{P\PYZus{}H1}\PY{o}{+}\PY{p}{(}\PY{l+m+mi}{1}\PY{o}{\PYZhy{}}\PY{n}{pi}\PY{p}{)}\PY{o}{*}\PY{n}{biclusters}\PY{p}{[}\PY{n}{bic1}\PY{p}{]}\PY{o}{.}\PY{n}{probability}\PY{o}{*}\PY{n}{biclusters}\PY{p}{[}\PY{n}{bic2}\PY{p}{]}\PY{o}{.}\PY{n}{probability}
                \PY{n}{newd}\PY{o}{=}\PY{n+nb}{float}\PY{p}{(}\PY{n}{scipy}\PY{o}{.}\PY{n}{misc}\PY{o}{.}\PY{n}{factorial}\PY{p}{(}\PY{n+nb}{len}\PY{p}{(}\PY{n}{newpoint}\PY{p}{)}\PY{o}{\PYZhy{}}\PY{l+m+mi}{1}\PY{p}{)}\PY{p}{)}\PY{o}{*}\PY{n}{alpha}\PY{o}{+}\PY{n}{biclusters}\PY{p}{[}\PY{n}{bic1}\PY{p}{]}\PY{o}{.}\PY{n}{d}\PY{o}{*}\PY{n}{biclusters}\PY{p}{[}\PY{n}{bic2}\PY{p}{]}\PY{o}{.}\PY{n}{d}
                \PY{c}{\PYZsh{}construct the new culster using the information above}
                \PY{n}{newbic} \PY{o}{=} \PY{n}{bicluster}\PY{p}{(}\PY{n}{point}\PY{o}{=}\PY{n}{newpoint}\PY{p}{,} \PY{n}{left}\PY{o}{=}\PY{n}{biclusters}\PY{p}{[}\PY{n}{bic1}\PY{p}{]}\PY{p}{,} \PY{n}{right}\PY{o}{=}\PY{n}{biclusters}\PY{p}{[}\PY{n}{bic2}\PY{p}{]}\PY{p}{,} \PY{n}{probability}\PY{o}{=}\PY{n}{newprob}\PY{p}{,} \PY{n}{d}\PY{o}{=}\PY{n}{newd}\PY{p}{,}  \PY{n+nb}{id} \PY{o}{=} \PY{n}{currentclusted}\PY{p}{)} 
                
                \PY{n}{currentclusted} \PY{o}{\PYZhy{}}\PY{o}{=} \PY{l+m+mi}{1}\PY{c}{\PYZsh{}update the id for the next cluster}
                \PY{c}{\PYZsh{}delete the two old clusters}
                \PY{k}{del} \PY{n}{biclusters}\PY{p}{[}\PY{n}{bic2}\PY{p}{]} 
                \PY{k}{del} \PY{n}{biclusters}\PY{p}{[}\PY{n}{bic1}\PY{p}{]}
                
                \PY{c}{\PYZsh{}add the new cluster into the biclusters}
                \PY{n}{biclusters}\PY{o}{.}\PY{n}{append}\PY{p}{(}\PY{n}{newbic}\PY{p}{)}
                \PY{c}{\PYZsh{}record the clustering result}
                \PY{n}{clusters} \PY{o}{=} \PY{p}{[}\PY{n}{yezi}\PY{p}{(}\PY{n}{biclusters}\PY{p}{[}\PY{n}{i}\PY{p}{]}\PY{p}{)} \PY{k}{for} \PY{n}{i} \PY{o+ow}{in} \PY{n+nb}{range}\PY{p}{(}\PY{n+nb}{len}\PY{p}{(}\PY{n}{biclusters}\PY{p}{)}\PY{p}{)}\PY{p}{]} 
                
            \PY{k}{return} \PY{n}{biclusters}\PY{p}{,}\PY{n}{clusters}
\end{Verbatim}

    \begin{longtable}[c]{@{}l@{}}
\toprule\addlinespace
3. Testing
\\\addlinespace
\bottomrule
\end{longtable}

In the unit test, I mainly test four things:

\begin{itemize}
\item
  whether the number of clusters is larger than or equal to 1;
\item
  whether the number of clusters is smaller than or equal to the number
  of data points;
\item
  whether the algorithm can produce one cluster when all data points are
  exactly the same;
\item
  whether the algorithm can produce one cluster when there is only one
  data point.
\end{itemize}

    \begin{Verbatim}[commandchars=\\\{\}]
{\color{incolor}In [{\color{incolor}}]:} \PY{o}{!} py.test
\end{Verbatim}

    \begin{longtable}[c]{@{}l@{}}
\toprule\addlinespace
4. Optimization
\\\addlinespace
\bottomrule
\end{longtable}

    In this part, I would focus on the efficiency of the code(accuracy will
be discussed later).

In order to test the efficiency, I generate a sample data set(which is
drawn from multinomial distribution):

    \begin{Verbatim}[commandchars=\\\{\}]
{\color{incolor}In [{\color{incolor}6}]:} \PY{n}{sample}\PY{o}{=}\PY{n}{np}\PY{o}{.}\PY{n}{array}\PY{p}{(}
               \PY{p}{[}\PY{p}{[}\PY{l+m+mi}{1}\PY{p}{,} \PY{l+m+mi}{0}\PY{p}{,} \PY{l+m+mi}{4}\PY{p}{,} \PY{l+m+mi}{5}\PY{p}{]}\PY{p}{,}
               \PY{p}{[}\PY{l+m+mi}{3}\PY{p}{,} \PY{l+m+mi}{2}\PY{p}{,} \PY{l+m+mi}{2}\PY{p}{,} \PY{l+m+mi}{3}\PY{p}{]}\PY{p}{,}
               \PY{p}{[}\PY{l+m+mi}{0}\PY{p}{,} \PY{l+m+mi}{3}\PY{p}{,} \PY{l+m+mi}{6}\PY{p}{,} \PY{l+m+mi}{1}\PY{p}{]}\PY{p}{,}
               \PY{p}{[}\PY{l+m+mi}{0}\PY{p}{,} \PY{l+m+mi}{2}\PY{p}{,} \PY{l+m+mi}{3}\PY{p}{,} \PY{l+m+mi}{5}\PY{p}{]}\PY{p}{,}
               \PY{p}{[}\PY{l+m+mi}{1}\PY{p}{,} \PY{l+m+mi}{0}\PY{p}{,} \PY{l+m+mi}{7}\PY{p}{,} \PY{l+m+mi}{2}\PY{p}{]}\PY{p}{,}
               \PY{p}{[}\PY{l+m+mi}{2}\PY{p}{,} \PY{l+m+mi}{0}\PY{p}{,} \PY{l+m+mi}{6}\PY{p}{,} \PY{l+m+mi}{2}\PY{p}{]}\PY{p}{,}
               \PY{p}{[}\PY{l+m+mi}{0}\PY{p}{,} \PY{l+m+mi}{0}\PY{p}{,} \PY{l+m+mi}{5}\PY{p}{,} \PY{l+m+mi}{5}\PY{p}{]}\PY{p}{,}
               \PY{p}{[}\PY{l+m+mi}{2}\PY{p}{,} \PY{l+m+mi}{1}\PY{p}{,} \PY{l+m+mi}{4}\PY{p}{,} \PY{l+m+mi}{3}\PY{p}{]}\PY{p}{,}
               \PY{p}{[}\PY{l+m+mi}{3}\PY{p}{,} \PY{l+m+mi}{4}\PY{p}{,} \PY{l+m+mi}{1}\PY{p}{,} \PY{l+m+mi}{2}\PY{p}{]}\PY{p}{,}
               \PY{p}{[}\PY{l+m+mi}{1}\PY{p}{,} \PY{l+m+mi}{3}\PY{p}{,} \PY{l+m+mi}{3}\PY{p}{,} \PY{l+m+mi}{3}\PY{p}{]}\PY{p}{]}\PY{p}{)}
\end{Verbatim}

    First, measure the time of function \textbf{bcluser} using the sample
data set:

    \begin{Verbatim}[commandchars=\\\{\}]
{\color{incolor}In [{\color{incolor}7}]:} \PY{o}{\PYZpc{}}\PY{k}{timeit} \PY{o}{\PYZhy{}}\PY{n}{n} \PY{l+m+mi}{100} \PY{n}{k}\PY{p}{,}\PY{n}{l}\PY{o}{=}\PY{n}{bcluster}\PY{p}{(}\PY{n}{sample}\PY{p}{,}\PY{n}{marginal\PYZus{}likelihood\PYZus{}DW}\PY{p}{)} 
\end{Verbatim}

    \begin{Verbatim}[commandchars=\\\{\}]
100 loops, best of 3: 23.1 ms per loop
    \end{Verbatim}

    To improve efficiency, I try two method: \textbf{numba} and
\textbf{cython}.

    \subsubsection{4.1 Numba}\label{numba}

Numba gives the power to speed up your applications with high
performance functions written directly in Python with just a few
annotations. The numba version of the function is shown below:

    \begin{Verbatim}[commandchars=\\\{\}]
{\color{incolor}In [{\color{incolor}8}]:} \PY{k+kn}{from} \PY{n+nn}{numba} \PY{k+kn}{import} \PY{n}{jit}\PY{p}{,} \PY{n}{int32}\PY{p}{,} \PY{n}{int64}\PY{p}{,} \PY{n}{float32}\PY{p}{,} \PY{n}{float64} 
        \PY{k+kn}{import} \PY{n+nn}{scipy.misc}
        \PY{k+kn}{from} \PY{n+nn}{numbapro} \PY{k+kn}{import} \PY{n}{vectorize}
        
        \PY{n+nd}{@jit}
        
        \PY{k}{def} \PY{n+nf}{yezi}\PY{p}{(}\PY{n}{clust}\PY{p}{)}\PY{p}{:}
            \PY{k}{if} \PY{n}{clust}\PY{o}{.}\PY{n}{left} \PY{o}{==} \PY{n+nb+bp}{None} \PY{o+ow}{and} \PY{n}{clust}\PY{o}{.}\PY{n}{right} \PY{o}{==} \PY{n+nb+bp}{None} \PY{p}{:}
                \PY{k}{return} \PY{p}{[}\PY{n}{clust}\PY{o}{.}\PY{n}{id}\PY{p}{]}
            \PY{k}{return} \PY{n}{yezi}\PY{p}{(}\PY{n}{clust}\PY{o}{.}\PY{n}{left}\PY{p}{)} \PY{o}{+} \PY{n}{yezi}\PY{p}{(}\PY{n}{clust}\PY{o}{.}\PY{n}{right}\PY{p}{)}
        
        \PY{k}{class} \PY{n+nc}{bicluster}\PY{p}{:}
            \PY{k}{def} \PY{n+nf}{\PYZus{}\PYZus{}init\PYZus{}\PYZus{}}\PY{p}{(}\PY{n+nb+bp}{self}\PY{p}{,} \PY{n}{point}\PY{p}{,} \PY{n}{left}\PY{o}{=}\PY{n+nb+bp}{None}\PY{p}{,}\PY{n}{right}\PY{o}{=}\PY{n+nb+bp}{None}\PY{p}{,}\PY{n}{probability}\PY{o}{=}\PY{n+nb+bp}{None}\PY{p}{,}\PY{n}{d}\PY{o}{=}\PY{n+nb+bp}{None}\PY{p}{,}\PY{n+nb}{id}\PY{o}{=}\PY{n+nb+bp}{None}\PY{p}{)}\PY{p}{:}
                \PY{n+nb+bp}{self}\PY{o}{.}\PY{n}{left} \PY{o}{=} \PY{n}{left}
                \PY{n+nb+bp}{self}\PY{o}{.}\PY{n}{right} \PY{o}{=} \PY{n}{right}  
                \PY{n+nb+bp}{self}\PY{o}{.}\PY{n}{point} \PY{o}{=} \PY{n}{point}   
                \PY{n+nb+bp}{self}\PY{o}{.}\PY{n}{id} \PY{o}{=} \PY{n+nb}{id}     
                \PY{n+nb+bp}{self}\PY{o}{.}\PY{n}{probability} \PY{o}{=} \PY{n}{probability}
                \PY{n+nb+bp}{self}\PY{o}{.}\PY{n}{d}\PY{o}{=}\PY{n}{d}
        
        \PY{k}{def} \PY{n+nf}{marginal\PYZus{}likelihood\PYZus{}DW}\PY{p}{(}\PY{n}{points}\PY{p}{)}\PY{p}{:}
            \PY{n}{p}\PY{o}{=}\PY{n+nb}{len}\PY{p}{(}\PY{n}{points}\PY{p}{[}\PY{l+m+mi}{0}\PY{p}{]}\PY{p}{)}
            \PY{n}{n}\PY{o}{=}\PY{n+nb}{len}\PY{p}{(}\PY{n}{points}\PY{p}{)}
            \PY{c}{\PYZsh{}parameters for the prior distribution}
            \PY{n}{alpha}\PY{o}{=}\PY{p}{[}\PY{l+m+mf}{1.0}\PY{o}{/}\PY{n}{p}\PY{p}{]}\PY{o}{*}\PY{n}{p}
            \PY{n}{alpha0}\PY{o}{=}\PY{n}{np}\PY{o}{.}\PY{n}{sum}\PY{p}{(}\PY{n}{alpha}\PY{p}{)}
            
            \PY{k}{def} \PY{n+nf}{temp}\PY{p}{(}\PY{n}{yi}\PY{p}{)}\PY{p}{:}
                \PY{n+nb}{sum}\PY{o}{=}\PY{l+m+mf}{1.0}
                \PY{k}{for} \PY{n}{i} \PY{o+ow}{in} \PY{n+nb}{range}\PY{p}{(}\PY{l+m+mi}{0}\PY{p}{,}\PY{n}{p}\PY{p}{)}\PY{p}{:}
                    \PY{n+nb}{sum}\PY{o}{=}\PY{n+nb}{sum}\PY{o}{*}\PY{n}{gamma}\PY{p}{(}\PY{n}{alpha}\PY{p}{[}\PY{n}{i}\PY{p}{]}\PY{o}{+}\PY{n}{yi}\PY{p}{[}\PY{n}{i}\PY{p}{]}\PY{p}{)}\PY{o}{/}\PY{p}{(}\PY{n}{gamma}\PY{p}{(}\PY{n}{yi}\PY{p}{[}\PY{n}{i}\PY{p}{]}\PY{o}{+}\PY{l+m+mi}{1}\PY{p}{)}\PY{o}{*}\PY{n}{gamma}\PY{p}{(}\PY{n}{alpha}\PY{p}{[}\PY{n}{i}\PY{p}{]}\PY{p}{)}\PY{p}{)}
                \PY{k}{return} \PY{n+nb}{sum}
            \PY{c}{\PYZsh{}calculate the marginal probability}
            \PY{n}{mar\PYZus{}like}\PY{o}{=}\PY{n+nb}{reduce}\PY{p}{(}\PY{k}{lambda} \PY{n}{x}\PY{p}{,}\PY{n}{y}\PY{p}{:} \PY{n}{x}\PY{o}{*}\PY{n}{y}\PY{p}{,} \PY{n+nb}{map}\PY{p}{(}\PY{n}{temp}\PY{p}{,}\PY{n}{points}\PY{p}{)}\PY{p}{)}\PY{o}{*}\PY{p}{(}\PY{n}{gamma}\PY{p}{(}\PY{n}{p}\PY{o}{+}\PY{l+m+mi}{1}\PY{p}{)}\PY{o}{*}\PY{n}{gamma}\PY{p}{(}\PY{n}{alpha0}\PY{p}{)}\PY{o}{/}\PY{n+nb}{float}\PY{p}{(}\PY{n}{gamma}\PY{p}{(}\PY{n}{alpha0}\PY{o}{+}\PY{n}{p}\PY{p}{)}\PY{p}{)}\PY{p}{)}\PY{o}{*}\PY{o}{*}\PY{n}{n}
            
            \PY{k}{return} \PY{n}{mar\PYZus{}like}
        
        \PY{k}{def} \PY{n+nf}{bcluster\PYZus{}numba}\PY{p}{(}\PY{n}{data}\PY{p}{,}\PY{n}{function}\PY{p}{)} \PY{p}{:}
            \PY{n}{dim}\PY{o}{=}\PY{n+nb}{len}\PY{p}{(}\PY{n}{data}\PY{p}{[}\PY{l+m+mi}{0}\PY{p}{]}\PY{p}{)}\PY{c}{\PYZsh{}get the dimension of the data points}
            \PY{n}{alpha}\PY{o}{=}\PY{l+m+mi}{3}
            \PY{n}{biclusters} \PY{o}{=} \PY{p}{[} \PY{n}{bicluster}\PY{p}{(}\PY{n}{point} \PY{o}{=} \PY{p}{[}\PY{n}{data}\PY{p}{[}\PY{n}{i}\PY{p}{]}\PY{p}{]}\PY{p}{,} \PY{n+nb}{id} \PY{o}{=} \PY{n}{i} \PY{p}{,}\PY{n}{probability}\PY{o}{=}\PY{l+m+mf}{0.0000001}\PY{p}{,}\PY{n}{d}\PY{o}{=}\PY{n}{alpha}\PY{p}{)} \PY{k}{for} \PY{n}{i} \PY{o+ow}{in} \PY{n+nb}{range}\PY{p}{(}\PY{n+nb}{len}\PY{p}{(}\PY{n}{data}\PY{p}{)}\PY{p}{)}\PY{p}{]} \PY{c}{\PYZsh{}initialize: each point is a cluster}
            \PY{n}{flag} \PY{o}{=} \PY{n+nb+bp}{None}\PY{p}{;}\PY{c}{\PYZsh{}record the id of the pair of clusters when merging}
            \PY{n}{currentclusted} \PY{o}{=} \PY{o}{\PYZhy{}}\PY{l+m+mi}{1} \PY{c}{\PYZsh{}id for the new cluster}
            
            \PY{c}{\PYZsh{}when there is only one data point in the data set}
            \PY{k}{if} \PY{n+nb}{len}\PY{p}{(}\PY{n}{biclusters}\PY{p}{)} \PY{o}{==} \PY{l+m+mi}{1}\PY{p}{:}
                \PY{n}{clusters} \PY{o}{=} \PY{p}{[}\PY{n}{yezi}\PY{p}{(}\PY{n}{biclusters}\PY{p}{[}\PY{n}{i}\PY{p}{]}\PY{p}{)} \PY{k}{for} \PY{n}{i} \PY{o+ow}{in} \PY{n+nb}{range}\PY{p}{(}\PY{n+nb}{len}\PY{p}{(}\PY{n}{biclusters}\PY{p}{)}\PY{p}{)}\PY{p}{]} 
                \PY{k}{return} \PY{n}{biclusters}\PY{p}{,}\PY{n}{clusters}
            
            \PY{c}{\PYZsh{}when there are more than one data points in the data set}
            \PY{k}{while}\PY{p}{(}\PY{n+nb}{len}\PY{p}{(}\PY{n}{biclusters}\PY{p}{)} \PY{o}{\PYZgt{}} \PY{l+m+mi}{1}\PY{p}{)} \PY{p}{:} 
                \PY{n}{max\PYZus{}prob} \PY{o}{=} \PY{l+m+mi}{0}\PY{p}{;} \PY{c}{\PYZsh{}initializa the maximum r\PYZus{}k}
                \PY{n}{biclusters\PYZus{}len} \PY{o}{=} \PY{n+nb}{len}\PY{p}{(}\PY{n}{biclusters}\PY{p}{)}\PY{c}{\PYZsh{}number of the current clusters}
                
                \PY{c}{\PYZsh{}find the pair of clusters that has the largest probability of merged hypothesis}
                \PY{k}{for} \PY{n}{i} \PY{o+ow}{in} \PY{n+nb}{range}\PY{p}{(}\PY{n}{biclusters\PYZus{}len}\PY{o}{\PYZhy{}}\PY{l+m+mi}{1}\PY{p}{)} \PY{p}{:}
                    \PY{k}{for} \PY{n}{j} \PY{o+ow}{in} \PY{n+nb}{range}\PY{p}{(}\PY{n}{i} \PY{o}{+} \PY{l+m+mi}{1}\PY{p}{,} \PY{n}{biclusters\PYZus{}len}\PY{p}{)} \PY{p}{:}
                        
                        \PY{n}{temp\PYZus{}cluster\PYZus{}points}\PY{o}{=} \PY{n}{biclusters}\PY{p}{[}\PY{n}{i}\PY{p}{]}\PY{o}{.}\PY{n}{point} \PY{o}{+} \PY{n}{biclusters}\PY{p}{[}\PY{n}{j}\PY{p}{]}\PY{o}{.}\PY{n}{point}\PY{c}{\PYZsh{}combine the two clusters into one cluster}
                        \PY{n}{P\PYZus{}H1}\PY{o}{=}\PY{n}{function}\PY{p}{(}\PY{n}{temp\PYZus{}cluster\PYZus{}points}\PY{p}{)}\PY{c}{\PYZsh{}calculate the  probability that the points of the two clusters under H\PYZus{}1 hypothesis }
                        \PY{c}{\PYZsh{}update the prior that all data points belong to one cluster}
                        \PY{n}{pi}\PY{o}{=}\PY{n+nb}{float}\PY{p}{(}\PY{n}{scipy}\PY{o}{.}\PY{n}{misc}\PY{o}{.}\PY{n}{factorial}\PY{p}{(}\PY{n+nb}{len}\PY{p}{(}\PY{n}{temp\PYZus{}cluster\PYZus{}points}\PY{p}{)}\PY{o}{\PYZhy{}}\PY{l+m+mi}{1}\PY{p}{)}\PY{p}{)}\PY{o}{*}\PY{n}{alpha}\PY{o}{/}\PY{p}{(}\PY{n+nb}{float}\PY{p}{(}\PY{n}{scipy}\PY{o}{.}\PY{n}{misc}\PY{o}{.}\PY{n}{factorial}\PY{p}{(}\PY{n+nb}{len}\PY{p}{(}\PY{n}{temp\PYZus{}cluster\PYZus{}points}\PY{p}{)}\PY{o}{\PYZhy{}}\PY{l+m+mi}{1}\PY{p}{)}\PY{p}{)}\PY{o}{*}\PY{n}{alpha}\PY{o}{+}\PY{n}{biclusters}\PY{p}{[}\PY{n}{i}\PY{p}{]}\PY{o}{.}\PY{n}{d}\PY{o}{*}\PY{n}{biclusters}\PY{p}{[}\PY{n}{j}\PY{p}{]}\PY{o}{.}\PY{n}{d}\PY{p}{)}
                        \PY{n}{marginal\PYZus{}prob}\PY{o}{=}\PY{n}{pi}\PY{o}{*}\PY{n}{P\PYZus{}H1}\PY{o}{+}\PY{p}{(}\PY{l+m+mi}{1}\PY{o}{\PYZhy{}}\PY{n}{pi}\PY{p}{)}\PY{o}{*}\PY{n}{biclusters}\PY{p}{[}\PY{n}{i}\PY{p}{]}\PY{o}{.}\PY{n}{probability}\PY{o}{*}\PY{n}{biclusters}\PY{p}{[}\PY{n}{j}\PY{p}{]}\PY{o}{.}\PY{n}{probability}\PY{c}{\PYZsh{}calculate the marginal probabilty of data in tree T\PYZus{}k}
                        \PY{n}{r} \PY{o}{=} \PY{n}{pi}\PY{o}{*}\PY{n}{P\PYZus{}H1}\PY{o}{/}\PY{n}{marginal\PYZus{}prob}\PY{c}{\PYZsh{}calculate the posterior probability of merged hypothesis using bayes rule}
                        \PY{k}{if} \PY{n}{r} \PY{o}{\PYZgt{}} \PY{n}{max\PYZus{}prob} \PY{p}{:}
                            \PY{n}{max\PYZus{}prob} \PY{o}{=} \PY{n}{r}
                            \PY{n}{flag} \PY{o}{=} \PY{p}{(}\PY{n}{i}\PY{p}{,}\PY{n}{j}\PY{p}{)}
                            
                \PY{c}{\PYZsh{}when to stop clustering     }
                \PY{k}{if} \PY{n}{max\PYZus{}prob}\PY{o}{\PYZlt{}}\PY{l+m+mf}{0.5}\PY{p}{:} 
                    \PY{k}{break}
                    
                \PY{n}{bic1}\PY{p}{,}\PY{n}{bic2} \PY{o}{=} \PY{n}{flag} \PY{c}{\PYZsh{}update the flag}
                
                \PY{n}{newpoint} \PY{o}{=} \PY{n}{biclusters}\PY{p}{[}\PY{n}{bic1}\PY{p}{]}\PY{o}{.}\PY{n}{point} \PY{o}{+} \PY{n}{biclusters}\PY{p}{[}\PY{n}{bic2}\PY{p}{]}\PY{o}{.}\PY{n}{point} \PY{c}{\PYZsh{}combine the points of two clusters into the new cluster}
                \PY{n}{P\PYZus{}H1}\PY{o}{=}\PY{n}{function}\PY{p}{(}\PY{n}{newpoint}\PY{p}{)}
                \PY{c}{\PYZsh{}calculate the corresponding marginal probability for the new cluster}
                \PY{n}{newprob}\PY{o}{=}\PY{n}{pi}\PY{o}{*}\PY{n}{P\PYZus{}H1}\PY{o}{+}\PY{p}{(}\PY{l+m+mi}{1}\PY{o}{\PYZhy{}}\PY{n}{pi}\PY{p}{)}\PY{o}{*}\PY{n}{biclusters}\PY{p}{[}\PY{n}{bic1}\PY{p}{]}\PY{o}{.}\PY{n}{probability}\PY{o}{*}\PY{n}{biclusters}\PY{p}{[}\PY{n}{bic2}\PY{p}{]}\PY{o}{.}\PY{n}{probability}
                \PY{n}{newd}\PY{o}{=}\PY{n+nb}{float}\PY{p}{(}\PY{n}{scipy}\PY{o}{.}\PY{n}{misc}\PY{o}{.}\PY{n}{factorial}\PY{p}{(}\PY{n+nb}{len}\PY{p}{(}\PY{n}{newpoint}\PY{p}{)}\PY{o}{\PYZhy{}}\PY{l+m+mi}{1}\PY{p}{)}\PY{p}{)}\PY{o}{*}\PY{n}{alpha}\PY{o}{+}\PY{n}{biclusters}\PY{p}{[}\PY{n}{bic1}\PY{p}{]}\PY{o}{.}\PY{n}{d}\PY{o}{*}\PY{n}{biclusters}\PY{p}{[}\PY{n}{bic2}\PY{p}{]}\PY{o}{.}\PY{n}{d}
                \PY{c}{\PYZsh{}construct the new culster using the information above}
                \PY{n}{newbic} \PY{o}{=} \PY{n}{bicluster}\PY{p}{(}\PY{n}{point}\PY{o}{=}\PY{n}{newpoint}\PY{p}{,} \PY{n}{left}\PY{o}{=}\PY{n}{biclusters}\PY{p}{[}\PY{n}{bic1}\PY{p}{]}\PY{p}{,} \PY{n}{right}\PY{o}{=}\PY{n}{biclusters}\PY{p}{[}\PY{n}{bic2}\PY{p}{]}\PY{p}{,} \PY{n}{probability}\PY{o}{=}\PY{n}{newprob}\PY{p}{,} \PY{n}{d}\PY{o}{=}\PY{n}{newd}\PY{p}{,}  \PY{n+nb}{id} \PY{o}{=} \PY{n}{currentclusted}\PY{p}{)} 
                
                \PY{n}{currentclusted} \PY{o}{\PYZhy{}}\PY{o}{=} \PY{l+m+mi}{1}\PY{c}{\PYZsh{}update the id for the next cluster}
                \PY{c}{\PYZsh{}delete the two old clusters}
                \PY{k}{del} \PY{n}{biclusters}\PY{p}{[}\PY{n}{bic2}\PY{p}{]} 
                \PY{k}{del} \PY{n}{biclusters}\PY{p}{[}\PY{n}{bic1}\PY{p}{]}
                
                \PY{c}{\PYZsh{}add the new cluster into the biclusters}
                \PY{n}{biclusters}\PY{o}{.}\PY{n}{append}\PY{p}{(}\PY{n}{newbic}\PY{p}{)}
                \PY{c}{\PYZsh{}record the clustering result}
                \PY{n}{clusters} \PY{o}{=} \PY{p}{[}\PY{n}{yezi}\PY{p}{(}\PY{n}{biclusters}\PY{p}{[}\PY{n}{i}\PY{p}{]}\PY{p}{)} \PY{k}{for} \PY{n}{i} \PY{o+ow}{in} \PY{n+nb}{range}\PY{p}{(}\PY{n+nb}{len}\PY{p}{(}\PY{n}{biclusters}\PY{p}{)}\PY{p}{)}\PY{p}{]} 
                
            \PY{k}{return} \PY{n}{biclusters}\PY{p}{,}\PY{n}{clusters}
\end{Verbatim}

    \begin{Verbatim}[commandchars=\\\{\}]
{\color{incolor}In [{\color{incolor}9}]:} \PY{o}{\PYZpc{}}\PY{k}{timeit} \PY{o}{\PYZhy{}}\PY{n}{n} \PY{l+m+mi}{100} \PY{n}{k}\PY{p}{,}\PY{n}{l}\PY{o}{=}\PY{n}{bcluster\PYZus{}numba}\PY{p}{(}\PY{n}{sample}\PY{p}{,}\PY{n}{marginal\PYZus{}likelihood\PYZus{}DW}\PY{p}{)}
\end{Verbatim}

    \begin{Verbatim}[commandchars=\\\{\}]
100 loops, best of 3: 25.1 ms per loop
    \end{Verbatim}

    The performance of numba version doesn't greatly reduce the time.

    \subsubsection{4.2 Cython}\label{cython}

The Cython language is a superset of the Python language that
additionally supports calling C functions and declaring C types on
variables and class attributes. This allows the compiler to generate
very efficient C code from Cython code.

    \begin{Verbatim}[commandchars=\\\{\}]
{\color{incolor}In [{\color{incolor}11}]:} \PY{o}{\PYZpc{}}\PY{k}{load\PYZus{}ext} \PY{n}{cythonmagic}
\end{Verbatim}

    \begin{Verbatim}[commandchars=\\\{\}]
{\color{incolor}In [{\color{incolor}12}]:} \PY{o}{\PYZpc{}\PYZpc{}}\PY{k}{cython}
         \PY{k+kn}{import} \PY{n+nn}{numpy} \PY{k+kn}{as} \PY{n+nn}{np}
         \PY{n}{cimport} \PY{n}{cython}
         \PY{n}{cimport} \PY{n}{numpy} \PY{k}{as} \PY{n}{np}
         \PY{k+kn}{from} \PY{n+nn}{scipy.special} \PY{k+kn}{import} \PY{n}{gamma} 
         \PY{k+kn}{import} \PY{n+nn}{scipy}
         \PY{k+kn}{import} \PY{n+nn}{scipy.misc}
         
         \PY{n+nd}{@cython.boundscheck}\PY{p}{(}\PY{n+nb+bp}{False}\PY{p}{)}
         \PY{n+nd}{@cython.wraparound}\PY{p}{(}\PY{n+nb+bp}{False}\PY{p}{)}
         \PY{n+nd}{@cython.profile}\PY{p}{(}\PY{n+nb+bp}{False}\PY{p}{)}
         
         \PY{n}{cdef} \PY{n}{marginal\PYZus{}likelihood\PYZus{}DW\PYZus{}cython}\PY{p}{(}\PY{n}{np}\PY{o}{.}\PY{n}{ndarray}\PY{p}{[}\PY{n+nb}{long}\PY{p}{,}\PY{n}{ndim}\PY{o}{=}\PY{l+m+mi}{2}\PY{p}{]} \PY{n}{points}\PY{p}{)}\PY{p}{:}
             \PY{n}{cdef} \PY{n+nb}{int} \PY{n}{p}\PY{o}{=}\PY{n+nb}{len}\PY{p}{(}\PY{n}{points}\PY{p}{[}\PY{l+m+mi}{0}\PY{p}{]}\PY{p}{)}
             \PY{n}{cdef} \PY{n+nb}{int} \PY{n}{n}\PY{o}{=}\PY{n+nb}{len}\PY{p}{(}\PY{n}{points}\PY{p}{)}
             \PY{n}{cdef} \PY{n}{np}\PY{o}{.}\PY{n}{ndarray} \PY{n}{alpha}\PY{o}{=}\PY{n}{np}\PY{o}{.}\PY{n}{repeat}\PY{p}{(}\PY{l+m+mf}{1.0}\PY{o}{/}\PY{n}{p}\PY{p}{,}\PY{n}{p}\PY{p}{)}
             \PY{n}{cdef} \PY{n}{double} \PY{n}{alpha0}\PY{o}{=}\PY{n}{np}\PY{o}{.}\PY{n}{sum}\PY{p}{(}\PY{n}{alpha}\PY{p}{)}
             \PY{n}{cdef} \PY{n}{double}  \PY{n}{mar\PYZus{}like}
             
             \PY{k}{def} \PY{n+nf}{temp}\PY{p}{(}\PY{n}{np}\PY{o}{.}\PY{n}{ndarray}\PY{p}{[}\PY{n+nb}{long}\PY{p}{,}\PY{n}{ndim}\PY{o}{=}\PY{l+m+mi}{1}\PY{p}{]} \PY{n}{yi}\PY{p}{)}\PY{p}{:}
                 \PY{n}{cdef} \PY{n+nb}{int} \PY{n}{i}
                 \PY{n}{cdef} \PY{n}{double} \PY{n+nb}{sum}\PY{o}{=}\PY{l+m+mf}{1.0}
                 \PY{k}{for} \PY{n}{i} \PY{o+ow}{in} \PY{n+nb}{range}\PY{p}{(}\PY{l+m+mi}{0}\PY{p}{,}\PY{n}{p}\PY{p}{)}\PY{p}{:}
                     \PY{n+nb}{sum}\PY{o}{=}\PY{n+nb}{sum}\PY{o}{*}\PY{n}{gamma}\PY{p}{(}\PY{n}{alpha}\PY{p}{[}\PY{n}{i}\PY{p}{]}\PY{o}{+}\PY{n}{yi}\PY{p}{[}\PY{n}{i}\PY{p}{]}\PY{p}{)}\PY{o}{/}\PY{p}{(}\PY{n}{gamma}\PY{p}{(}\PY{n}{yi}\PY{p}{[}\PY{n}{i}\PY{p}{]}\PY{o}{+}\PY{l+m+mi}{1}\PY{p}{)}\PY{o}{*}\PY{n}{gamma}\PY{p}{(}\PY{n}{alpha}\PY{p}{[}\PY{n}{i}\PY{p}{]}\PY{p}{)}\PY{p}{)}
                 \PY{k}{return} \PY{n+nb}{sum}
             
             \PY{n}{mar\PYZus{}like}\PY{o}{=}\PY{n+nb}{reduce}\PY{p}{(}\PY{k}{lambda} \PY{n}{x}\PY{p}{,}\PY{n}{y}\PY{p}{:} \PY{n}{x}\PY{o}{*}\PY{n}{y}\PY{p}{,} \PY{n+nb}{map}\PY{p}{(}\PY{n}{temp}\PY{p}{,}\PY{n}{points}\PY{p}{)}\PY{p}{)}\PY{o}{*}\PY{p}{(}\PY{n}{gamma}\PY{p}{(}\PY{n}{p}\PY{o}{+}\PY{l+m+mi}{1}\PY{p}{)}\PY{o}{*}\PY{n}{gamma}\PY{p}{(}\PY{n}{alpha0}\PY{p}{)}\PY{o}{/}\PY{n+nb}{float}\PY{p}{(}\PY{n}{gamma}\PY{p}{(}\PY{n}{alpha0}\PY{o}{+}\PY{n}{p}\PY{p}{)}\PY{p}{)}\PY{p}{)}\PY{o}{*}\PY{o}{*}\PY{n}{n}
             
             \PY{k}{return} \PY{n}{mar\PYZus{}like}
         
         \PY{k}{class} \PY{n+nc}{bicluster}\PY{p}{:}
             \PY{k}{def} \PY{n+nf}{\PYZus{}\PYZus{}init\PYZus{}\PYZus{}}\PY{p}{(}\PY{n+nb+bp}{self}\PY{p}{,} \PY{n}{point}\PY{p}{,} \PY{n}{left}\PY{o}{=}\PY{n+nb+bp}{None}\PY{p}{,}\PY{n}{right}\PY{o}{=}\PY{n+nb+bp}{None}\PY{p}{,}\PY{n}{probability}\PY{o}{=}\PY{n+nb+bp}{None}\PY{p}{,}\PY{n}{d}\PY{o}{=}\PY{n+nb+bp}{None}\PY{p}{,}\PY{n+nb}{id}\PY{o}{=}\PY{n+nb+bp}{None}\PY{p}{)}\PY{p}{:}
                 \PY{n+nb+bp}{self}\PY{o}{.}\PY{n}{left} \PY{o}{=} \PY{n}{left}
                 \PY{n+nb+bp}{self}\PY{o}{.}\PY{n}{right} \PY{o}{=} \PY{n}{right}  
                 \PY{n+nb+bp}{self}\PY{o}{.}\PY{n}{point} \PY{o}{=} \PY{n}{point}   
                 \PY{n+nb+bp}{self}\PY{o}{.}\PY{n}{id} \PY{o}{=} \PY{n+nb}{id}     
                 \PY{n+nb+bp}{self}\PY{o}{.}\PY{n}{probability} \PY{o}{=} \PY{n}{probability}
                 \PY{n+nb+bp}{self}\PY{o}{.}\PY{n}{d}\PY{o}{=}\PY{n}{d}
             
         \PY{k}{def} \PY{n+nf}{yezi}\PY{p}{(}\PY{n}{clust}\PY{p}{)}\PY{p}{:}
             \PY{k}{if} \PY{n}{clust}\PY{o}{.}\PY{n}{left} \PY{o}{==} \PY{n+nb+bp}{None} \PY{o+ow}{and} \PY{n}{clust}\PY{o}{.}\PY{n}{right} \PY{o}{==} \PY{n+nb+bp}{None} \PY{p}{:}
                 \PY{k}{return} \PY{p}{[}\PY{n}{clust}\PY{o}{.}\PY{n}{id}\PY{p}{]}
             \PY{k}{return} \PY{n}{yezi}\PY{p}{(}\PY{n}{clust}\PY{o}{.}\PY{n}{left}\PY{p}{)} \PY{o}{+} \PY{n}{yezi}\PY{p}{(}\PY{n}{clust}\PY{o}{.}\PY{n}{right}\PY{p}{)}
         
         
         
         \PY{k}{def} \PY{n+nf}{bcluster\PYZus{}cython}\PY{p}{(}\PY{n}{np}\PY{o}{.}\PY{n}{ndarray}\PY{p}{[}\PY{n+nb}{long}\PY{p}{,}\PY{n}{ndim}\PY{o}{=}\PY{l+m+mi}{2}\PY{p}{]} \PY{n}{data}\PY{p}{)} \PY{p}{:}
            
             \PY{n}{cdef} \PY{n+nb}{int} \PY{n}{dim}\PY{o}{=}\PY{n+nb}{len}\PY{p}{(}\PY{n}{data}\PY{p}{[}\PY{l+m+mi}{0}\PY{p}{]}\PY{p}{)}
             \PY{n}{cdef} \PY{n+nb}{int} \PY{n}{alpha}\PY{o}{=}\PY{l+m+mi}{3}
             \PY{n}{cdef} \PY{n+nb}{int} \PY{n}{currentclusted} \PY{o}{=} \PY{o}{\PYZhy{}}\PY{l+m+mi}{1} 
             \PY{n}{cdef} \PY{n}{double} \PY{n}{P\PYZus{}H1}\PY{p}{,} \PY{n}{r}\PY{p}{,} \PY{n}{marginal\PYZus{}prob}\PY{p}{,} \PY{n}{pi}\PY{p}{,} \PY{n}{max\PYZus{}prob}\PY{p}{,} \PY{n}{newprob}\PY{p}{,} \PY{n}{newd}
             \PY{n}{cdef} \PY{n+nb}{int} \PY{n}{bic1}\PY{p}{,} \PY{n}{bic2}\PY{p}{,} \PY{n}{biclusters\PYZus{}len}
             \PY{n}{cdef} \PY{n+nb}{tuple} \PY{n}{flag} \PY{o}{=} \PY{n+nb+bp}{None}
             \PY{n}{cdef} \PY{n}{np}\PY{o}{.}\PY{n}{ndarray} \PY{n}{temppoints}\PY{p}{,} \PY{n}{newpoints}
             
             \PY{n}{biclusters} \PY{o}{=} \PY{p}{[} \PY{n}{bicluster}\PY{p}{(}\PY{n}{point} \PY{o}{=} \PY{n}{data}\PY{p}{[}\PY{n}{i}\PY{p}{]}\PY{p}{,} \PY{n+nb}{id} \PY{o}{=} \PY{n}{i} \PY{p}{,}\PY{n}{probability}\PY{o}{=}\PY{l+m+mf}{0.001}\PY{p}{,}\PY{n}{d}\PY{o}{=}\PY{n}{alpha}\PY{p}{)} \PY{k}{for} \PY{n}{i} \PY{o+ow}{in} \PY{n+nb}{range}\PY{p}{(}\PY{n+nb}{len}\PY{p}{(}\PY{n}{data}\PY{p}{)}\PY{p}{)}\PY{p}{]} \PY{c}{\PYZsh{}initialize: each point is a cluster}
             
             \PY{k}{while}\PY{p}{(}\PY{n+nb}{len}\PY{p}{(}\PY{n}{biclusters}\PY{p}{)} \PY{o}{\PYZgt{}} \PY{l+m+mi}{1}\PY{p}{)} \PY{p}{:} 
                 \PY{n}{max\PYZus{}prob} \PY{o}{=} \PY{l+m+mi}{0}\PY{p}{;} 
                 \PY{n}{biclusters\PYZus{}len} \PY{o}{=} \PY{n+nb}{len}\PY{p}{(}\PY{n}{biclusters}\PY{p}{)}
                 \PY{k}{for} \PY{n}{i} \PY{o+ow}{in} \PY{n+nb}{range}\PY{p}{(}\PY{n}{biclusters\PYZus{}len}\PY{o}{\PYZhy{}}\PY{l+m+mi}{1}\PY{p}{)} \PY{p}{:}
                     \PY{k}{for} \PY{n}{j} \PY{o+ow}{in} \PY{n+nb}{range}\PY{p}{(}\PY{n}{i} \PY{o}{+} \PY{l+m+mi}{1}\PY{p}{,} \PY{n}{biclusters\PYZus{}len}\PY{p}{)} \PY{p}{:}
                         
                         \PY{c}{\PYZsh{}calculate P\PYZus{}H1: MC method}
                         \PY{n}{temppoints}\PY{o}{=} \PY{n}{np}\PY{o}{.}\PY{n}{vstack}\PY{p}{(}\PY{p}{(}\PY{n}{biclusters}\PY{p}{[}\PY{n}{i}\PY{p}{]}\PY{o}{.}\PY{n}{point} \PY{p}{,} \PY{n}{biclusters}\PY{p}{[}\PY{n}{j}\PY{p}{]}\PY{o}{.}\PY{n}{point}\PY{p}{)}\PY{p}{)}
                         
                         \PY{c}{\PYZsh{}P\PYZus{}H1=marginal\PYZus{}likelihood\PYZus{}NIW(temp\PYZus{}cluster\PYZus{}points)}
                         \PY{n}{P\PYZus{}H1}\PY{o}{=}\PY{n}{marginal\PYZus{}likelihood\PYZus{}DW\PYZus{}cython}\PY{p}{(}\PY{n}{temppoints}\PY{p}{)}
                         \PY{n}{pi}\PY{o}{=}\PY{n+nb}{float}\PY{p}{(}\PY{n}{scipy}\PY{o}{.}\PY{n}{misc}\PY{o}{.}\PY{n}{factorial}\PY{p}{(}\PY{n+nb}{len}\PY{p}{(}\PY{n}{temppoints}\PY{p}{)}\PY{o}{\PYZhy{}}\PY{l+m+mi}{1}\PY{p}{)}\PY{p}{)}\PY{o}{*}\PY{n}{alpha}\PY{o}{/}\PY{p}{(}\PY{n+nb}{float}\PY{p}{(}\PY{n}{scipy}\PY{o}{.}\PY{n}{misc}\PY{o}{.}\PY{n}{factorial}\PY{p}{(}\PY{n+nb}{len}\PY{p}{(}\PY{n}{temppoints}\PY{p}{)}\PY{o}{\PYZhy{}}\PY{l+m+mi}{1}\PY{p}{)}\PY{p}{)}\PY{o}{*}\PY{n}{alpha}\PY{o}{+}\PY{n}{biclusters}\PY{p}{[}\PY{n}{i}\PY{p}{]}\PY{o}{.}\PY{n}{d}\PY{o}{*}\PY{n}{biclusters}\PY{p}{[}\PY{n}{j}\PY{p}{]}\PY{o}{.}\PY{n}{d}\PY{p}{)}
                         \PY{n}{marginal\PYZus{}prob}\PY{o}{=}\PY{n}{pi}\PY{o}{*}\PY{n}{P\PYZus{}H1}\PY{o}{+}\PY{p}{(}\PY{l+m+mi}{1}\PY{o}{\PYZhy{}}\PY{n}{pi}\PY{p}{)}\PY{o}{*}\PY{n}{biclusters}\PY{p}{[}\PY{n}{i}\PY{p}{]}\PY{o}{.}\PY{n}{probability}\PY{o}{*}\PY{n}{biclusters}\PY{p}{[}\PY{n}{j}\PY{p}{]}\PY{o}{.}\PY{n}{probability}
                         \PY{n}{r} \PY{o}{=} \PY{n}{pi}\PY{o}{*}\PY{n}{P\PYZus{}H1}\PY{o}{/}\PY{n}{marginal\PYZus{}prob}
                         \PY{k}{if} \PY{n}{r} \PY{o}{\PYZgt{}} \PY{n}{max\PYZus{}prob} \PY{p}{:}
                             \PY{n}{max\PYZus{}prob} \PY{o}{=} \PY{n}{r}
                             \PY{n}{flag} \PY{o}{=} \PY{p}{(}\PY{n}{i}\PY{p}{,}\PY{n}{j}\PY{p}{)}
                 
                 \PY{k}{if} \PY{n}{max\PYZus{}prob}\PY{o}{\PYZlt{}}\PY{l+m+mf}{0.5}\PY{p}{:} 
                     \PY{k}{break}
                     
                 \PY{n}{bic1}\PY{p}{,}\PY{n}{bic2} \PY{o}{=} \PY{n}{flag} 
                 
                 \PY{n}{newpoints} \PY{o}{=} \PY{n}{np}\PY{o}{.}\PY{n}{vstack}\PY{p}{(}\PY{p}{(}\PY{n}{biclusters}\PY{p}{[}\PY{n}{bic1}\PY{p}{]}\PY{o}{.}\PY{n}{point} \PY{p}{,} \PY{n}{biclusters}\PY{p}{[}\PY{n}{bic2}\PY{p}{]}\PY{o}{.}\PY{n}{point}\PY{p}{)}\PY{p}{)} \PY{c}{\PYZsh{}combine the points of two clusters into the new cluster}
                 
                 \PY{c}{\PYZsh{}P\PYZus{}H1=marginal\PYZus{}likelihood\PYZus{}NIW(newpoint)}
                 \PY{n}{P\PYZus{}H1}\PY{o}{=}\PY{n}{marginal\PYZus{}likelihood\PYZus{}DW\PYZus{}cython}\PY{p}{(}\PY{n}{newpoints}\PY{p}{)}
                 
                 \PY{n}{newprob}\PY{o}{=}\PY{n}{pi}\PY{o}{*}\PY{n}{P\PYZus{}H1}\PY{o}{+}\PY{p}{(}\PY{l+m+mi}{1}\PY{o}{\PYZhy{}}\PY{n}{pi}\PY{p}{)}\PY{o}{*}\PY{n}{biclusters}\PY{p}{[}\PY{n}{bic1}\PY{p}{]}\PY{o}{.}\PY{n}{probability}\PY{o}{*}\PY{n}{biclusters}\PY{p}{[}\PY{n}{bic2}\PY{p}{]}\PY{o}{.}\PY{n}{probability}
                 \PY{n}{newd}\PY{o}{=}\PY{n+nb}{float}\PY{p}{(}\PY{n}{scipy}\PY{o}{.}\PY{n}{misc}\PY{o}{.}\PY{n}{factorial}\PY{p}{(}\PY{n+nb}{len}\PY{p}{(}\PY{n}{newpoints}\PY{p}{)}\PY{o}{\PYZhy{}}\PY{l+m+mi}{1}\PY{p}{)}\PY{p}{)}\PY{o}{*}\PY{n}{alpha}\PY{o}{+}\PY{n}{biclusters}\PY{p}{[}\PY{n}{bic1}\PY{p}{]}\PY{o}{.}\PY{n}{d}\PY{o}{*}\PY{n}{biclusters}\PY{p}{[}\PY{n}{bic2}\PY{p}{]}\PY{o}{.}\PY{n}{d}
                 \PY{n}{newbic} \PY{o}{=} \PY{n}{bicluster}\PY{p}{(}\PY{n}{point}\PY{o}{=}\PY{n}{newpoints}\PY{p}{,} \PY{n}{left}\PY{o}{=}\PY{n}{biclusters}\PY{p}{[}\PY{n}{bic1}\PY{p}{]}\PY{p}{,} \PY{n}{right}\PY{o}{=}\PY{n}{biclusters}\PY{p}{[}\PY{n}{bic2}\PY{p}{]}\PY{p}{,} \PY{n}{probability}\PY{o}{=}\PY{n}{newprob}\PY{p}{,} \PY{n}{d}\PY{o}{=}\PY{n}{newd}\PY{p}{,}  \PY{n+nb}{id} \PY{o}{=} \PY{n}{currentclusted}\PY{p}{)} 
                 \PY{n}{currentclusted} \PY{o}{\PYZhy{}}\PY{o}{=} \PY{l+m+mi}{1}
                 
                 \PY{k}{del} \PY{n}{biclusters}\PY{p}{[}\PY{n}{bic2}\PY{p}{]} 
                 \PY{k}{del} \PY{n}{biclusters}\PY{p}{[}\PY{n}{bic1}\PY{p}{]}
                 \PY{n}{biclusters}\PY{o}{.}\PY{n}{append}\PY{p}{(}\PY{n}{newbic}\PY{p}{)}
                 \PY{n}{clusters} \PY{o}{=} \PY{p}{[}\PY{n}{yezi}\PY{p}{(}\PY{n}{biclusters}\PY{p}{[}\PY{n}{i}\PY{p}{]}\PY{p}{)} \PY{k}{for} \PY{n}{i} \PY{o+ow}{in} \PY{n+nb}{range}\PY{p}{(}\PY{n+nb}{len}\PY{p}{(}\PY{n}{biclusters}\PY{p}{)}\PY{p}{)}\PY{p}{]} 
                 
             \PY{k}{return} \PY{n}{biclusters}\PY{p}{,}\PY{n}{clusters}
\end{Verbatim}

    \begin{Verbatim}[commandchars=\\\{\}]
{\color{incolor}In [{\color{incolor}13}]:} \PY{o}{\PYZpc{}}\PY{k}{timeit} \PY{o}{\PYZhy{}}\PY{n}{n} \PY{l+m+mi}{100} \PY{n}{k}\PY{p}{,}\PY{n}{l}\PY{o}{=}\PY{n}{bcluster\PYZus{}cython}\PY{p}{(}\PY{n}{sample}\PY{p}{)} 
\end{Verbatim}

    \begin{Verbatim}[commandchars=\\\{\}]
100 loops, best of 3: 22.6 ms per loop
    \end{Verbatim}

    The performance of cython version is not desirable; it doesn't greatly
reduce the time. It may be due to the class bicluster. I couldn't find a
way to transform the class bicluster into C type, which might be the
main reason.

    \begin{longtable}[c]{@{}l@{}}
\toprule\addlinespace
5. Application and Comparison
\\\addlinespace
\bottomrule
\end{longtable}

\subsubsection{5.1 Application to Simulated Data
Set}\label{application-to-simulated-data-set}

I simulate two data sets:

\begin{itemize}
\itemsep1pt\parskip0pt\parsep0pt
\item
  \textbf{Normal}: I simulate three data points from $N_2((0,0)',0.5I)$,
  4 data points from $N_2((2,2)',0.5I)$, and 3 data points from
  $N_2((8,8)',0.5I)$. Then combine them into one data set called
  sample\_NIW. This data set is supposed to be clustered into 3
  clusters.
\end{itemize}

    \begin{Verbatim}[commandchars=\\\{\}]
{\color{incolor}In [{\color{incolor}14}]:} \PY{c}{\PYZsh{}sample multinormal data}
         \PY{n}{mean2}\PY{o}{=}\PY{p}{(}\PY{l+m+mi}{0}\PY{p}{,}\PY{l+m+mi}{0}\PY{p}{)}
         \PY{n}{cov2}\PY{o}{=}\PY{l+m+mf}{0.5}\PY{o}{*}\PY{n}{np}\PY{o}{.}\PY{n}{eye}\PY{p}{(}\PY{l+m+mi}{2}\PY{p}{)}
         \PY{n}{np}\PY{o}{.}\PY{n}{random}\PY{o}{.}\PY{n}{seed}\PY{p}{(}\PY{l+m+mi}{4}\PY{p}{)}
         \PY{n}{data2}\PY{o}{=}\PY{n}{np}\PY{o}{.}\PY{n}{random}\PY{o}{.}\PY{n}{multivariate\PYZus{}normal}\PY{p}{(}\PY{n}{mean2}\PY{p}{,}\PY{n}{cov2}\PY{p}{,}\PY{l+m+mi}{3}\PY{p}{)}
         \PY{n}{mean3}\PY{o}{=}\PY{p}{(}\PY{l+m+mi}{2}\PY{p}{,}\PY{l+m+mi}{2}\PY{p}{)}
         \PY{n}{cov3}\PY{o}{=}\PY{l+m+mf}{0.5}\PY{o}{*}\PY{n}{np}\PY{o}{.}\PY{n}{eye}\PY{p}{(}\PY{l+m+mi}{2}\PY{p}{)}
         \PY{n}{np}\PY{o}{.}\PY{n}{random}\PY{o}{.}\PY{n}{seed}\PY{p}{(}\PY{l+m+mi}{5}\PY{p}{)}
         \PY{n}{data3}\PY{o}{=}\PY{n}{np}\PY{o}{.}\PY{n}{random}\PY{o}{.}\PY{n}{multivariate\PYZus{}normal}\PY{p}{(}\PY{n}{mean3}\PY{p}{,}\PY{n}{cov3}\PY{p}{,}\PY{l+m+mi}{4}\PY{p}{)}
         \PY{n}{mean4}\PY{o}{=}\PY{p}{(}\PY{l+m+mi}{8}\PY{p}{,}\PY{l+m+mi}{8}\PY{p}{)}
         \PY{n}{cov4}\PY{o}{=}\PY{l+m+mf}{0.5}\PY{o}{*}\PY{n}{np}\PY{o}{.}\PY{n}{eye}\PY{p}{(}\PY{l+m+mi}{2}\PY{p}{)}
         \PY{n}{np}\PY{o}{.}\PY{n}{random}\PY{o}{.}\PY{n}{seed}\PY{p}{(}\PY{l+m+mi}{6}\PY{p}{)}
         \PY{n}{data4}\PY{o}{=}\PY{n}{np}\PY{o}{.}\PY{n}{random}\PY{o}{.}\PY{n}{multivariate\PYZus{}normal}\PY{p}{(}\PY{n}{mean4}\PY{p}{,}\PY{n}{cov4}\PY{p}{,}\PY{l+m+mi}{3}\PY{p}{)}
         \PY{n}{sample\PYZus{}NIW}\PY{o}{=}\PY{n}{np}\PY{o}{.}\PY{n}{concatenate}\PY{p}{(}\PY{p}{(}\PY{n}{data3}\PY{p}{,} \PY{n}{data4}\PY{p}{,}\PY{n}{data2}\PY{p}{)}\PY{p}{,} \PY{n}{axis}\PY{o}{=}\PY{l+m+mi}{0}\PY{p}{)}
\end{Verbatim}

    \begin{itemize}
\itemsep1pt\parskip0pt\parsep0pt
\item
  \textbf{Multinomial}: Data set sample\_DW contains 3 identical data
  points {[}3, 3, 3, 3, 3, 3{]}, and other 3 identical data points {[}9,
  9, 0, 0, 0, 0{]}. This data set is supposed to be clustered into 2
  clusters.
\end{itemize}

    \begin{Verbatim}[commandchars=\\\{\}]
{\color{incolor}In [{\color{incolor}15}]:} \PY{c}{\PYZsh{}sample multinomial data}
         
         \PY{n}{d1}\PY{o}{=}\PY{p}{[}\PY{p}{[}\PY{l+m+mi}{3}\PY{p}{,} \PY{l+m+mi}{3}\PY{p}{,} \PY{l+m+mi}{3}\PY{p}{,} \PY{l+m+mi}{3}\PY{p}{,} \PY{l+m+mi}{3}\PY{p}{,} \PY{l+m+mi}{3}\PY{p}{]}\PY{p}{,}\PY{p}{[}\PY{l+m+mi}{3}\PY{p}{,} \PY{l+m+mi}{3}\PY{p}{,} \PY{l+m+mi}{3}\PY{p}{,} \PY{l+m+mi}{3}\PY{p}{,} \PY{l+m+mi}{3}\PY{p}{,} \PY{l+m+mi}{3}\PY{p}{]}\PY{p}{,}\PY{p}{[}\PY{l+m+mi}{3}\PY{p}{,} \PY{l+m+mi}{3}\PY{p}{,} \PY{l+m+mi}{3}\PY{p}{,} \PY{l+m+mi}{3}\PY{p}{,} \PY{l+m+mi}{3}\PY{p}{,} \PY{l+m+mi}{3}\PY{p}{]}\PY{p}{]}
         \PY{n}{d2}\PY{o}{=}\PY{p}{[}\PY{p}{[}\PY{l+m+mi}{9}\PY{p}{,} \PY{l+m+mi}{9}\PY{p}{,} \PY{l+m+mi}{0}\PY{p}{,} \PY{l+m+mi}{0}\PY{p}{,} \PY{l+m+mi}{0}\PY{p}{,} \PY{l+m+mi}{0}\PY{p}{]}\PY{p}{,}\PY{p}{[}\PY{l+m+mi}{9}\PY{p}{,} \PY{l+m+mi}{9}\PY{p}{,} \PY{l+m+mi}{0}\PY{p}{,} \PY{l+m+mi}{0}\PY{p}{,} \PY{l+m+mi}{0}\PY{p}{,} \PY{l+m+mi}{0}\PY{p}{]}\PY{p}{,}\PY{p}{[}\PY{l+m+mi}{9}\PY{p}{,} \PY{l+m+mi}{9}\PY{p}{,} \PY{l+m+mi}{0}\PY{p}{,} \PY{l+m+mi}{0}\PY{p}{,} \PY{l+m+mi}{0}\PY{p}{,} \PY{l+m+mi}{0}\PY{p}{]}\PY{p}{]}
         \PY{n}{sample\PYZus{}DW}\PY{o}{=}\PY{n}{np}\PY{o}{.}\PY{n}{concatenate}\PY{p}{(}\PY{p}{(}\PY{n}{d1}\PY{p}{,} \PY{n}{d2}\PY{p}{)}\PY{p}{,} \PY{n}{axis}\PY{o}{=}\PY{l+m+mi}{0}\PY{p}{)}
\end{Verbatim}

    \subsubsection{5.2 Comparison}\label{comparison}

\paragraph{5.2.1 Bayesian Hierarchical
Clustering}\label{bayesian-hierarchical-clustering}

    \begin{Verbatim}[commandchars=\\\{\}]
{\color{incolor}In [{\color{incolor}16}]:} \PY{c}{\PYZsh{}Clustering result using normal data set:}
         \PY{n}{k1}\PY{p}{,}\PY{n}{l1}\PY{o}{=}\PY{n}{bcluster}\PY{p}{(}\PY{n}{sample\PYZus{}NIW}\PY{p}{,}\PY{n}{marginal\PYZus{}likelihood\PYZus{}NIW}\PY{p}{)} 
         \PY{n}{l1}
\end{Verbatim}

            \begin{Verbatim}[commandchars=\\\{\}]
{\color{outcolor}Out[{\color{outcolor}16}]:} [[7, 8], [6, 4, 5, 0, 1, 2, 3, 9]]
\end{Verbatim}
        
    \begin{Verbatim}[commandchars=\\\{\}]
{\color{incolor}In [{\color{incolor}17}]:} \PY{c}{\PYZsh{}Clustering result using multinomial data set:}
         \PY{n}{k2}\PY{p}{,}\PY{n}{l2}\PY{o}{=}\PY{n}{bcluster}\PY{p}{(}\PY{n}{sample\PYZus{}DW}\PY{p}{,}\PY{n}{marginal\PYZus{}likelihood\PYZus{}DW}\PY{p}{)} 
         \PY{n}{l2}
\end{Verbatim}

            \begin{Verbatim}[commandchars=\\\{\}]
{\color{outcolor}Out[{\color{outcolor}17}]:} [[1], [2], [0, 5, 3, 4]]
\end{Verbatim}
        
    For the normal data set sample\_DW, function bcluster divides the data
points into 2 clusters, and for the multinomial data set sample\_DW, it
divides the data points into three clusters. Though bayesian
hierarchical clustering indeed correctly clusters part of the points,
the overall performances are not good.

The main reason for the undesirable performance comes from the choice of
probability distribution and its prior(including parameters). In the
above two clustering processes, even though I use the correct
probability distributions(multinomial and normal), it is hard to decide
whether the priors are correct. In this case, I use conjugate priors for
the sake of simplicity, but conjugate priors may not be the best one.
What's more, the values of the parameters in the prior distributions are
difficult to choose without additional information.

More importantly, in some situations we don't know how the data is
generated, so the choice of probability distribution is also a
considerable problem.

    \paragraph{5.2.2 Traditional Hierarchical
Clustering}\label{traditional-hierarchical-clustering}

I first compare the result with traditional hierarchical clustering
based on Euclidean distance.(The function using traditional hierarchical
clustering is shown below)

    \begin{Verbatim}[commandchars=\\\{\}]
{\color{incolor}In [{\color{incolor}18}]:} \PY{c}{\PYZsh{}define class tbicluster}
         \PY{k}{class} \PY{n+nc}{tbicluster}\PY{p}{:}
             \PY{k}{def} \PY{n+nf}{\PYZus{}\PYZus{}init\PYZus{}\PYZus{}}\PY{p}{(}\PY{n+nb+bp}{self}\PY{p}{,} \PY{n}{vec}\PY{p}{,} \PY{n}{left}\PY{o}{=}\PY{n+nb+bp}{None}\PY{p}{,}\PY{n}{right}\PY{o}{=}\PY{n+nb+bp}{None}\PY{p}{,}\PY{n}{distance}\PY{o}{=}\PY{l+m+mf}{0.0}\PY{p}{,}\PY{n+nb}{id}\PY{o}{=}\PY{n+nb+bp}{None}\PY{p}{)}\PY{p}{:}
                 \PY{n+nb+bp}{self}\PY{o}{.}\PY{n}{left} \PY{o}{=} \PY{n}{left}
                 \PY{n+nb+bp}{self}\PY{o}{.}\PY{n}{right} \PY{o}{=} \PY{n}{right}  
                 \PY{n+nb+bp}{self}\PY{o}{.}\PY{n}{vec} \PY{o}{=} \PY{n}{vec}     
                 \PY{n+nb+bp}{self}\PY{o}{.}\PY{n}{id} \PY{o}{=} \PY{n+nb}{id}     
                 \PY{n+nb+bp}{self}\PY{o}{.}\PY{n}{distance} \PY{o}{=} \PY{n}{distance}
                 
         \PY{c}{\PYZsh{}culstering function            }
         \PY{k}{def} \PY{n+nf}{hcluster}\PY{p}{(}\PY{n}{data}\PY{p}{,}\PY{n}{n}\PY{p}{)} \PY{p}{:}
             
             \PY{k}{def} \PY{n+nf}{Euclidean\PYZus{}distance}\PY{p}{(}\PY{n}{vector1}\PY{p}{,}\PY{n}{vector2}\PY{p}{)}\PY{p}{:}
                 \PY{n}{length} \PY{o}{=} \PY{n+nb}{len}\PY{p}{(}\PY{n}{vector1}\PY{p}{)}
                 \PY{n}{TSum} \PY{o}{=} \PY{n+nb}{sum}\PY{p}{(}\PY{p}{[}\PY{n+nb}{pow}\PY{p}{(}\PY{p}{(}\PY{n}{vector1}\PY{p}{[}\PY{n}{i}\PY{p}{]} \PY{o}{\PYZhy{}} \PY{n}{vector2}\PY{p}{[}\PY{n}{i}\PY{p}{]}\PY{p}{)}\PY{p}{,}\PY{l+m+mi}{2}\PY{p}{)} \PY{k}{for} \PY{n}{i} \PY{o+ow}{in} \PY{n+nb}{range}\PY{p}{(}\PY{n+nb}{len}\PY{p}{(}\PY{n}{vector1}\PY{p}{)}\PY{p}{)}\PY{p}{]}\PY{p}{)}
                 \PY{n}{SSum} \PY{o}{=} \PY{n}{sqrt}\PY{p}{(}\PY{n}{TSum}\PY{p}{)}  
                 \PY{k}{return} \PY{n}{SSum}
             
             \PY{n}{biclusters} \PY{o}{=} \PY{p}{[} \PY{n}{tbicluster}\PY{p}{(}\PY{n}{vec} \PY{o}{=} \PY{n}{data}\PY{p}{[}\PY{n}{i}\PY{p}{]}\PY{p}{,} \PY{n+nb}{id} \PY{o}{=} \PY{n}{i} \PY{p}{)} \PY{k}{for} \PY{n}{i} \PY{o+ow}{in} \PY{n+nb}{range}\PY{p}{(}\PY{n+nb}{len}\PY{p}{(}\PY{n}{data}\PY{p}{)}\PY{p}{)} \PY{p}{]}
             \PY{n}{distances} \PY{o}{=} \PY{p}{\PYZob{}}\PY{p}{\PYZcb{}}
             \PY{n}{flag} \PY{o}{=} \PY{n+nb+bp}{None}\PY{p}{;}
             \PY{n}{currentclusted} \PY{o}{=} \PY{o}{\PYZhy{}}\PY{l+m+mi}{1}
             \PY{k}{while}\PY{p}{(}\PY{n+nb}{len}\PY{p}{(}\PY{n}{biclusters}\PY{p}{)} \PY{o}{\PYZgt{}} \PY{n}{n}\PY{p}{)} \PY{p}{:} 
                 \PY{n}{min\PYZus{}val} \PY{o}{=} \PY{l+m+mi}{999999999999}\PY{p}{;} 
                 \PY{n}{biclusters\PYZus{}len} \PY{o}{=} \PY{n+nb}{len}\PY{p}{(}\PY{n}{biclusters}\PY{p}{)}
                 \PY{k}{for} \PY{n}{i} \PY{o+ow}{in} \PY{n+nb}{range}\PY{p}{(}\PY{n}{biclusters\PYZus{}len}\PY{o}{\PYZhy{}}\PY{l+m+mi}{1}\PY{p}{)} \PY{p}{:}
                     \PY{k}{for} \PY{n}{j} \PY{o+ow}{in} \PY{n+nb}{range}\PY{p}{(}\PY{n}{i} \PY{o}{+} \PY{l+m+mi}{1}\PY{p}{,} \PY{n}{biclusters\PYZus{}len}\PY{p}{)} \PY{p}{:}
                         \PY{k}{if} \PY{n}{distances}\PY{o}{.}\PY{n}{get}\PY{p}{(}\PY{p}{(}\PY{n}{biclusters}\PY{p}{[}\PY{n}{i}\PY{p}{]}\PY{o}{.}\PY{n}{id}\PY{p}{,}\PY{n}{biclusters}\PY{p}{[}\PY{n}{j}\PY{p}{]}\PY{o}{.}\PY{n}{id}\PY{p}{)}\PY{p}{)} \PY{o}{==} \PY{n+nb+bp}{None}\PY{p}{:}
                             \PY{n}{distances}\PY{p}{[}\PY{p}{(}\PY{n}{biclusters}\PY{p}{[}\PY{n}{i}\PY{p}{]}\PY{o}{.}\PY{n}{id}\PY{p}{,}\PY{n}{biclusters}\PY{p}{[}\PY{n}{j}\PY{p}{]}\PY{o}{.}\PY{n}{id}\PY{p}{)}\PY{p}{]} \PY{o}{=} \PY{n}{Euclidean\PYZus{}distance}\PY{p}{(}\PY{n}{biclusters}\PY{p}{[}\PY{n}{i}\PY{p}{]}\PY{o}{.}\PY{n}{vec}\PY{p}{,}\PY{n}{biclusters}\PY{p}{[}\PY{n}{j}\PY{p}{]}\PY{o}{.}\PY{n}{vec}\PY{p}{)}
                         \PY{n}{d} \PY{o}{=} \PY{n}{distances}\PY{p}{[}\PY{p}{(}\PY{n}{biclusters}\PY{p}{[}\PY{n}{i}\PY{p}{]}\PY{o}{.}\PY{n}{id}\PY{p}{,}\PY{n}{biclusters}\PY{p}{[}\PY{n}{j}\PY{p}{]}\PY{o}{.}\PY{n}{id}\PY{p}{)}\PY{p}{]} 
                         \PY{k}{if} \PY{n}{d} \PY{o}{\PYZlt{}} \PY{n}{min\PYZus{}val} \PY{p}{:}
                             \PY{n}{min\PYZus{}val} \PY{o}{=} \PY{n}{d}
                             \PY{n}{flag} \PY{o}{=} \PY{p}{(}\PY{n}{i}\PY{p}{,}\PY{n}{j}\PY{p}{)}
                 \PY{n}{bic1}\PY{p}{,}\PY{n}{bic2} \PY{o}{=} \PY{n}{flag} 
                 \PY{n}{newvec} \PY{o}{=} \PY{p}{[}\PY{p}{(}\PY{n}{biclusters}\PY{p}{[}\PY{n}{bic1}\PY{p}{]}\PY{o}{.}\PY{n}{vec}\PY{p}{[}\PY{n}{i}\PY{p}{]} \PY{o}{+} \PY{n}{biclusters}\PY{p}{[}\PY{n}{bic2}\PY{p}{]}\PY{o}{.}\PY{n}{vec}\PY{p}{[}\PY{n}{i}\PY{p}{]}\PY{p}{)}\PY{o}{/}\PY{l+m+mi}{2} \PY{k}{for} \PY{n}{i} \PY{o+ow}{in} \PY{n+nb}{range}\PY{p}{(}\PY{n+nb}{len}\PY{p}{(}\PY{n}{biclusters}\PY{p}{[}\PY{n}{bic1}\PY{p}{]}\PY{o}{.}\PY{n}{vec}\PY{p}{)}\PY{p}{)}\PY{p}{]} 
                 \PY{n}{newbic} \PY{o}{=} \PY{n}{tbicluster}\PY{p}{(}\PY{n}{newvec}\PY{p}{,} \PY{n}{left}\PY{o}{=}\PY{n}{biclusters}\PY{p}{[}\PY{n}{bic1}\PY{p}{]}\PY{p}{,} \PY{n}{right}\PY{o}{=}\PY{n}{biclusters}\PY{p}{[}\PY{n}{bic2}\PY{p}{]}\PY{p}{,} \PY{n}{distance}\PY{o}{=}\PY{n}{min\PYZus{}val}\PY{p}{,} \PY{n+nb}{id} \PY{o}{=} \PY{n}{currentclusted}\PY{p}{)} 
                 \PY{n}{currentclusted} \PY{o}{\PYZhy{}}\PY{o}{=} \PY{l+m+mi}{1}
                 \PY{k}{del} \PY{n}{biclusters}\PY{p}{[}\PY{n}{bic2}\PY{p}{]} 
                 \PY{k}{del} \PY{n}{biclusters}\PY{p}{[}\PY{n}{bic1}\PY{p}{]}
                 \PY{n}{biclusters}\PY{o}{.}\PY{n}{append}\PY{p}{(}\PY{n}{newbic}\PY{p}{)}
                 \PY{n}{clusters} \PY{o}{=} \PY{p}{[}\PY{n}{yezi}\PY{p}{(}\PY{n}{biclusters}\PY{p}{[}\PY{n}{i}\PY{p}{]}\PY{p}{)} \PY{k}{for} \PY{n}{i} \PY{o+ow}{in} \PY{n+nb}{range}\PY{p}{(}\PY{n+nb}{len}\PY{p}{(}\PY{n}{biclusters}\PY{p}{)}\PY{p}{)}\PY{p}{]} 
             \PY{k}{return} \PY{n}{biclusters}\PY{p}{,}\PY{n}{clusters}
\end{Verbatim}

    The result for the two data sets are shown below(given the correct
number of clusters):

    \begin{Verbatim}[commandchars=\\\{\}]
{\color{incolor}In [{\color{incolor}19}]:} \PY{c}{\PYZsh{}Culstering result using normal data set:}
         \PY{n}{tk1}\PY{p}{,}\PY{n}{tl1}\PY{o}{=}\PY{n}{hcluster}\PY{p}{(}\PY{n}{sample\PYZus{}NIW}\PY{p}{,}\PY{l+m+mi}{3}\PY{p}{)} 
         \PY{n}{tl1}
\end{Verbatim}

            \begin{Verbatim}[commandchars=\\\{\}]
{\color{outcolor}Out[{\color{outcolor}19}]:} [[9, 7, 8], [1, 2, 0, 3], [6, 4, 5]]
\end{Verbatim}
        
    \begin{Verbatim}[commandchars=\\\{\}]
{\color{incolor}In [{\color{incolor}20}]:} \PY{c}{\PYZsh{}Clustering result using multinomial data set:}
         \PY{n}{tk2}\PY{p}{,}\PY{n}{tl2}\PY{o}{=}\PY{n}{hcluster}\PY{p}{(}\PY{n}{sample\PYZus{}DW}\PY{p}{,}\PY{l+m+mi}{2}\PY{p}{)} 
         \PY{n}{tl2}
\end{Verbatim}

            \begin{Verbatim}[commandchars=\\\{\}]
{\color{outcolor}Out[{\color{outcolor}20}]:} [[2, 0, 1], [5, 3, 4]]
\end{Verbatim}
        
    For both the normal data set sample\_NIW and the multinomial data set
sample\_DW, traditional hierarchical clustering correctly clusters them
into 3 and 2 groups with the right components.

Although this alhorithm works fine in this case, it still have some
concerns. The algorithm provides no guide to choosing the ``correct''
number of clusters or the level at which to prune the tree. It is often
difficult to know which distance metric to choose, especially for
structured data such as images or sequences. The traditional algorithm
does not define a probabilistic model of the data, so it is hard to ask
how ``good'' a clustering is, to compare to other models, to make
predictions and cluster new data into an existing hierarchy.

    \paragraph{5.2.3 K-means Clustering}\label{k-means-clustering}

Next, I use K-Means algorithm to cluster the two data sets. The K-Means
algorithm clusters data by trying to separate samples in n groups of
equal variance, minimizing a criterion known as the \textbf{inertia} or
within-cluster sum-of-squares. This algorithm requires the number of
clusters to be specified.

K-means is often referred to as Lloyd's algorithm. In basic terms, the
algorithm has three steps. The first step chooses the initial centroids,
with the most basic method being to choose k samples from the dataset X.
After initialization, K-means consists of looping between the two other
steps. The first step assigns each sample to its nearest centroid. The
second step creates new centroids by taking the mean value of all of the
samples assigned to each previous centroid. The difference between the
old and the new centroids are computed and the algorithm repeats these
last two steps until this value is less than a threshold. In other
words, it repeats until the centroids do not move significantly.

The function is shown below:

    \begin{Verbatim}[commandchars=\\\{\}]
{\color{incolor}In [{\color{incolor}21}]:} \PY{c}{\PYZsh{}kmeans}
         \PY{k+kn}{import} \PY{n+nn}{random}
         
         \PY{k}{def} \PY{n+nf}{kmeanscluster}\PY{p}{(}\PY{n}{X}\PY{p}{,} \PY{n}{K}\PY{p}{)}\PY{p}{:}
             
             \PY{k}{def} \PY{n+nf}{cluster\PYZus{}points}\PY{p}{(}\PY{n}{X}\PY{p}{,} \PY{n}{mu}\PY{p}{)}\PY{p}{:}
                 \PY{n}{clusters}  \PY{o}{=} \PY{p}{\PYZob{}}\PY{p}{\PYZcb{}}
                 \PY{k}{for} \PY{n}{x} \PY{o+ow}{in} \PY{n}{X}\PY{p}{:}
                     \PY{n}{bestmukey} \PY{o}{=} \PY{n+nb}{min}\PY{p}{(}\PY{p}{[}\PY{p}{(}\PY{n}{i}\PY{p}{[}\PY{l+m+mi}{0}\PY{p}{]}\PY{p}{,} \PY{n}{np}\PY{o}{.}\PY{n}{linalg}\PY{o}{.}\PY{n}{norm}\PY{p}{(}\PY{n}{x}\PY{o}{\PYZhy{}}\PY{n}{mu}\PY{p}{[}\PY{n}{i}\PY{p}{[}\PY{l+m+mi}{0}\PY{p}{]}\PY{p}{]}\PY{p}{)}\PY{p}{)}  \PY{k}{for} \PY{n}{i} \PY{o+ow}{in} \PY{n+nb}{enumerate}\PY{p}{(}\PY{n}{mu}\PY{p}{)}\PY{p}{]}\PY{p}{,} \PY{n}{key}\PY{o}{=}\PY{k}{lambda} \PY{n}{t}\PY{p}{:}\PY{n}{t}\PY{p}{[}\PY{l+m+mi}{1}\PY{p}{]}\PY{p}{)}\PY{p}{[}\PY{l+m+mi}{0}\PY{p}{]}
                     \PY{k}{try}\PY{p}{:}
                         \PY{n}{clusters}\PY{p}{[}\PY{n}{bestmukey}\PY{p}{]}\PY{o}{.}\PY{n}{append}\PY{p}{(}\PY{n}{x}\PY{p}{)}
                     \PY{k}{except} \PY{n+ne}{KeyError}\PY{p}{:}
                         \PY{n}{clusters}\PY{p}{[}\PY{n}{bestmukey}\PY{p}{]} \PY{o}{=} \PY{p}{[}\PY{n}{x}\PY{p}{]}
                 \PY{k}{return} \PY{n}{clusters}
         
             \PY{k}{def} \PY{n+nf}{reevaluate}\PY{p}{(}\PY{n}{mu}\PY{p}{,} \PY{n}{clusters}\PY{p}{)}\PY{p}{:}
                 \PY{n}{newmu} \PY{o}{=} \PY{p}{[}\PY{p}{]}
                 \PY{n}{keys} \PY{o}{=} \PY{n+nb}{sorted}\PY{p}{(}\PY{n}{clusters}\PY{o}{.}\PY{n}{keys}\PY{p}{(}\PY{p}{)}\PY{p}{)}
                 \PY{k}{for} \PY{n}{k} \PY{o+ow}{in} \PY{n}{keys}\PY{p}{:}
                     \PY{n}{newmu}\PY{o}{.}\PY{n}{append}\PY{p}{(}\PY{n}{np}\PY{o}{.}\PY{n}{mean}\PY{p}{(}\PY{n}{clusters}\PY{p}{[}\PY{n}{k}\PY{p}{]}\PY{p}{,} \PY{n}{axis} \PY{o}{=} \PY{l+m+mi}{0}\PY{p}{)}\PY{p}{)}
                 \PY{k}{return} \PY{n}{newmu}
          
             \PY{k}{def} \PY{n+nf}{ifconverged}\PY{p}{(}\PY{n}{mu}\PY{p}{,} \PY{n}{oldmu}\PY{p}{)}\PY{p}{:}
                 \PY{k}{return} \PY{n+nb}{set}\PY{p}{(}\PY{p}{[}\PY{n+nb}{tuple}\PY{p}{(}\PY{n}{a}\PY{p}{)} \PY{k}{for} \PY{n}{a} \PY{o+ow}{in} \PY{n}{mu}\PY{p}{]}\PY{p}{)} \PY{o}{==} \PY{n+nb}{set}\PY{p}{(}\PY{p}{[}\PY{n+nb}{tuple}\PY{p}{(}\PY{n}{a}\PY{p}{)} \PY{k}{for} \PY{n}{a} \PY{o+ow}{in} \PY{n}{oldmu}\PY{p}{]}\PY{p}{)}
         
             \PY{c}{\PYZsh{} Initialize to K random centers}
             \PY{n}{oldmu} \PY{o}{=} \PY{n}{random}\PY{o}{.}\PY{n}{sample}\PY{p}{(}\PY{n}{X}\PY{p}{,} \PY{n}{K}\PY{p}{)}
             \PY{n}{mu} \PY{o}{=} \PY{n}{random}\PY{o}{.}\PY{n}{sample}\PY{p}{(}\PY{n}{X}\PY{p}{,} \PY{n}{K}\PY{p}{)}
             \PY{k}{while} \PY{o+ow}{not} \PY{n}{ifconverged}\PY{p}{(}\PY{n}{mu}\PY{p}{,} \PY{n}{oldmu}\PY{p}{)}\PY{p}{:}
                 \PY{n}{oldmu} \PY{o}{=} \PY{n}{mu}
                 \PY{c}{\PYZsh{} Assign all points in X to clusters}
                 \PY{n}{kclusters} \PY{o}{=} \PY{n}{cluster\PYZus{}points}\PY{p}{(}\PY{n}{X}\PY{p}{,} \PY{n}{mu}\PY{p}{)}
                 \PY{c}{\PYZsh{} Reevaluate centers}
                 \PY{n}{mu} \PY{o}{=} \PY{n}{reevaluate}\PY{p}{(}\PY{n}{oldmu}\PY{p}{,} \PY{n}{kclusters}\PY{p}{)}
             \PY{k}{return} \PY{n}{mu}\PY{p}{,} \PY{n}{kclusters}
\end{Verbatim}

    The results for the two data sets are are not stable because of the
first step--choosing the initial centroids randomly. Sometimes this
algorithm could produce the correct clusters, while other times the
results are wrong.

    Besides, k-means suffers from various drawbacks:

\begin{itemize}
\item
  Inertia makes the assumption that clusters are convex and isotropic,
  which is not always the case. It responds poorly to elongated
  clusters, or manifolds with irregular shapes.
\item
  Inertia is not a normalized metric: we just know that lower values are
  better and zero is optimal. But in very high-dimensional spaces,
  Euclidean distances tend to become inflated.
\end{itemize}

    

    \begin{longtable}[c]{@{}l@{}}
\toprule\addlinespace
6. Future Work
\\\addlinespace
\bottomrule
\end{longtable}

\begin{itemize}
\item
  Efficiency: the optimization of bayesian hierarchical clustering has
  much to do with the data structure. In the function, I use class to
  define each cluster, but it is hard to be transformed into C. I will
  try to use a new data structure to define cluster, thus increasing
  efficiency.
\item
  Accuracy: as discuassed before, the inaccuracy of bayesian
  hierarchical clustering mainly derives from the choice of probability
  distribution and its prior. For the prior, I'll find some optimization
  algorithms to choose the best values for the parameters.
\end{itemize}

    


    % Add a bibliography block to the postdoc
    
    
    
    \end{document}
